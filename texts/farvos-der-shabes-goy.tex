% Corrected by: Masha Trius
% Email: triusmaria@gmail.com
% Text source: מעשהלעך און משלים פון נפתלי גראָס

\startcomponent oytser
\project reader

\starttext
\startchapter[title={פֿאַרוואָס דער שבת־גוי מיקאָלע איז געווען אַ ציוניסט}] [
	author={אויסקליבן פֿון נפֿתּלי גראָס},
]
אין שטעטל, צווישן די בחורים זענען געווען צווי שקצים, מיקאָלע און שטעפֿאַן. נישט קיין היצעלעס, חלילה, נישט קיין עזות־פּנימער. נאָר דווקא וואוילע שקצים. זיי האָבן גערעדט ייִדיש און געגאַנגען אָנגעטאָן ווי אַלע ייִדישע בחורים. איינער פֿון זיי, שטעפֿאַן, איז געווען אַ שוסטער־יונג, אַ געזעל, אַ האָרעפּאַשניק און ער האָט זיך געחבֿרט מיט די געזעלן, מיט די אַַרבעטער. דער אַנדערער, מיקאָלע, איז געווען אַ שטיקל אונטער־שמש, אַ שבת־גוי. ער האָט געהאָלפֿן אויסקערן דאָס בית־מדרש, אָנטראָגן וואַסער אין האַנטפֿאַס און שבת מאַכן פײַער אין אויוון.



\stopchapter
\stoptext

\stopcomponent
