% Typed by: Masha Trius
% Email: triusmaria@gmail.com
% Text source: מעשהלעך און משלים פון נפתּלי גראָס

\startcomponent oytser
\project reader

\starttext
\startchapter[title={פֿאַרוואָס דער שבת־גוי מיקאָלע איז געווען אַ ציוניסט}] [
	author={אויסקליבן פֿון נפֿתּלי גראָס},
]
אין שטעטל, צווישן די בחורים זענען געווען צווי שקצים, מיקאָלע און שטעפֿאַן. נישט קיין היצעלעס, חלילה, נישט קיין עזות־פּנימער. נאָר דווקא וואוילע שקצים. זיי האָבן גערעדט ייִדיש און געגאַנגען אָנגעטאָן ווי אַלע ייִדישע בחורים. איינער פֿון זיי, שטעפֿאַן, איז געווען אַ שוסטער־יונג, אַ געזעל, אַ האָרעפּאַשניק און ער האָט זיך געחבֿרט מיט די געזעלן, מיט די אַַרבעטער. דער אַנדערער, מיקאָלע, איז געווען אַ שטיקל אונטער־שמש, אַ שבת־גוי. ער האָט געהאָלפֿן אויסקערן דאָס בית־מדרש, אָנטראָגן וואַסער אין האַנטפֿאַס און שבת מאַכן פײַער אין אויוון.

מאַכט זיך אַמאָל, אַז פֿון דער גרויסער שטאָט, פֿון מינסק, קומען אַראָפּ אין שטעטל אַרײַן פּאַרטיי־מענטשן, אַגיטאַטאָרס, פֿײַערדיקע רעדנער, און אַרבעטן מעשים. זיי האָבן מיט זייערע רעדעס געמאַכט אַן איבערקערעניש אין שטעטל און צעטיילט דעם עולם אין צוויי לאַגערן. יעדער פֿון זיי האָט געוואָלט געווינען דעם עולם פֿאַר זײַן זאַך. האָט זיך געקערט וועלטן. טייל האָבן גערעדט פֿאַרן ייִדישן אַרבעטער בונד. זיי האָבן אָנגעוויזן ווי גוט סע וועט זײַן, ווען סע וועט הערשן סאָציאַליזם. אויס רײַכע און אויס אָרעמע!  אַלע וועלן דעמאָלט זײַן גלײַך. אַנדערע ווידער האָבן גערעדט וועגן ארץ ישׂראל און ציוניזם. זיי האָבן געטענהט, אַז ס׳איז צײַט, אַז ייִדן זאָלן זיך אומקערן אין אייגענעם לאַנד אַרײַן און זײַן אַ פֿאָלק מיט פֿעלקער צוגלײַך.

שטעפֿאַן, דער שוסטער־געזעל, האָט זיך אָנגעשלאָסן אין „בונד“. ער איז געוואָרן אַ געטרײַער פּאַרטיי־חבֿר און האָט געהאַלפֿן אָרגאַניזירן אַלע געזעלן אין שטעטל. מיקאָלען איז געפֿעלן דער ציוניזם. ער האָט געזען די אויסלייזונג פֿון די אונטערדריקטע אין דער פֿאַרווירקלעכונג פֿון ציוניזם.



\stopchapter
\stoptext

\stopcomponent
