% Corrected by: Marc Trius
% Email: derpayatz@gmail.com
% Text source: https://ocr.yiddishbookcenter.org/
% Date retrieved: 5/24/2020

\doublehyphendemerits=20000
\startcomponent dray-matones
\project reader

\starttext
\startchapter[title={דרײַ מתּנות}] [
	author={י. ל. פּרץ}
]

\startsection[title={אױבען באַ דער װאג}]

מאָל , מיט יאָר און דורות צוריק , אַין ערגיץ נטר געװ רען אַ יוד .

מילא , אַ יוד אַין נפטר געװאָרען — אוּיבּיג לעבּען קען מען ניט — טוסוט מען איהם זײן רעכט . . . בּרעעט מען איהם צו קכורת ישראל . . .

נאָך סתימת הגולל , דער ישם זאָגט קדיש — פליהט די נשמה אַרױף , צום משפט , צום בּית-דין של מעלה . . .

קומט זי אַרױף , הענט שױן האָרט פאַר'ן בּית-דין די װאָג , אוּיף װל ער מען װעגט די עברות און די מצװת . . .

קומט דעם כּר-מגן'ס סניגור , זײן געװעזענער יצר טוב — און שטעלט זיך מיט אַ קלאָר װײס זעקיל װי שנײ אין דער האַנד , בּײ דער װאָגשאָל מון דער רעכטער זײט . . .

קומט דעם -מנן'ס קטיגור — זײן געװעזענער יצר הרע , דער גע- װעזענער מסית ומדה — און שטעלט זיך , מיט אַ קױטיג זעקיל אַין דער האַנד , בּײ דער ראָג-שאָל אױף דער לינקער דיט . . .

אַין װײסען קלאָדען זעקיל זענען מצװת , אַין בּרודיק שאַרעען ע— קיל — עבירות . דט דער סניגור , פון שנײ-װײסען זעקיל אױף דער װאָגשל פון דער רעכטער זײט , מצװת — שמעקען זײ װי פארפומעס און ליכטען װי די יעטערנלעך אַין הימעל .

שיט דער קטיגור פון בּרודגען זעקיל אַרױס , אױף .דער װאָג-שאָל מן דער נקער האַנד , עבירות — זענען די , ניש פאַר אַײך געדאַכט , שװאַרץ װי ק יל און אַ רה האָבּען זײ — סאַרע-ראָדנע פעך און מאָלע .

שטעהט אַזױ די אָרימע נשמה און קוקט , און גאַפט — זי האָ זיך גאַר ניט געריכט , עס זאָל זײן אַזאַ הלוק צװישען „גוט" און „שלעט" ;

י. ל. סריל

דאָרט האָט זי גאַנץ אָפט בּײדע נישט דערקענט , אײנעם פאַר'ן אַנדערן גענומען .

און די װאָר-שאָלן הױבּען זיך סאַמעליך , אַתיף און אַראָבּ , אַמאָל די , אַמאָל יענע . . . און דאָס צינגעל בּײ דער װאָגישאָל אַין דער ה. ך ציטערט , נעהמט זיך אױף אַ האָר לינקס , אױף אַ האָר רעכטס . . .

טר אױף אַ האָר . . . און נישט מיט אַ מאָל ! אַ פּראָסטער יוד געװען , אָהן גרױסען „להכעיס" , ר אױך אָהן „קדוש השם-' . . . אַזױ — קלײנע מצוה'ליד , קלײנע עצה'ליד , שרעטעליך , שטױבּעליך . . . מײלמאָל — קױם מיט'ן אױג צו דערזעהן .

און פודעסטרעגען , אַז דאָס ציעעל רוקט זיך אױף אַ האָר רעכטס , הערט זיך דן ד עולמות דעליונים אַ קלאַע פון שמהה , פון נהת ; רואָ זיך עס אָבּער , הס ושלום , לינקס , לאָגט דך אַרױס אַ זיפץ פון ענות , בּין ער דערגרײכט צום כּסא הכבוד .

און ר אכים שיטען פּשעליך , מיט כּונה. אַ שרױט נאָך אַ שרױט , 'אַ שטױבּ נאָך אַ שטובּ , װי פּשוט'ע בּעל -בּתּים ליציטירען „אַתּה הראית" , אָ פּרוטה און נאָך אַ פרוטה . . .

טר אַ בּרונעם שעפּט זיך אײך אױס , ד זעקליך װערען לײדיק .

— שױחך ? פרעגט דער בּית-דין- מש — אױך אַ מלאך צװישען אלע

דעד יעדר טונ , װי דער שר הרע , דרעהען אױס די זעקליך קאַסױר

נישטאָ מעהר ! און דעמאָלט געהט דער שמש צו צום צינגעל זעהן , ור עס האָט זיך אָבּזנעשטעלט : דעכטס צי לינקס .

קוקט זנר און קוקט אַזױ , און זעהט אַזױנס , װאָס עס איז נאָך ניט געשעהען , זײד הימעל און ערד זענען בּעשפען געװאָרען . . .

— װאָס דױערט אַזױ לאַנג י פרעגט דער אנ-בּית-דין .

שטאַמזנלט דער שמש :

— גיך ! דאָס צינגעל שטעהט אין סאַמע-ראָדנע מיט ! . . .

די ענ-ות האָבּען גיך געװארגען מיט די מצװת .

— אַקוראַט ז פרעגט מען נאָך אַ מאָל פון בּות-דין לן מעל'ס טיש . קוקט זור נאָך אַ מאָל דער מש און ענטפערט :

אױן- דער האָר !

קאָט דיאָס בּית-דין — מלה אַ ישוב און גיט , נאָכ'ן ישוב הדעת

אַרױס אַ פּסק כהאי לשנא :

סאָלקסטימליכע געשיכטען .

„היות ד שאים װעגען נישט איבּער די מעלס טובים — קומט דער נשמה קײן גיהנם נישט . . .

,און פון צװײטען צד — װעגען ד מצװת נײטט מעהר פאַר די ענרות לי— קען מען איהר נישט עפענען די מױערן פון גן-עדן .

,ל כן נע-ונד זל זי זײן . . .

„זאָל זי אַרוממהען און דער מיט , צװישן הימעל און ע.ד — צין גאָט װעט זיך אָן דהר דערמאַנען און דערבּאַרמען זיך , און צודופק צו זיך מיט זײן גנאָד" .

און דער שמש נעהמט די נשמה און סיהרט ד אַרױס .

יאָמערט נעבּיך די נ מה און קלאָגט אױף איהר מזל .

— װאָס װײנסטו אַזױ ז רעגט ער זי: װעסט נישט האָבּען קײן נתת און רײד פון גן-עדן , װעסט דו דך אָבּער אױך נישט האָבּען קײן צער און יסורים פון גיהנם , — קרט !

הי נשמה אָבּער אָזט זיך נישט טרײסטען :

. בּעסער די גרעסטע יסורים , זאָגט זי , אײדער גאָר נישט ! גאָר נישט אַין שדעקליך !

קריגט ער אױף אַיהר , דער בּית-דין-שמש , רשנות , גיט ער אָיהה אַײן עצה :

„פליה , זאָגט ער אַיהר , גשה'לע , אַראָבּ און האַלט זיך נידריג אַרום דער לעבּעדיגער װלט . . .

„אין הימל אַרײן , זאָגט ער , קוק נישט . . . װאָרום װאָס װעסטו חו זעהן אין יענער זײט הימעל? נײערט שטערנליך ! און דאָס זענען ליכטיגע נור קאַלטע רואים , זײ װײסען פון קײן רהמנות נישט , זײ װעל זיך נישט מיהען פאַר דיר , זײ װאן גאָט אָן דיר נישט דערמאַנען . . .

„מיהען זיך האָר אַײן אָרימער פערראָגעלטער נשקך קענען נור די צדיקים פון גן-עדן... און זײ , הער נשמה'ל , — זײ האָבּען ליעבּ מתּנות.... מענע מתּנות . . . עס אַין מין — גיט ער צו בּיטער*ך — די טנע סון הײנטיגע צדיקים

„ה זע , נשמה'ש — זאָגט ער װײטער — נידריג אַרום דער לעבעדיגער װעלט און מעק זיך צו װי עס לעבּט זיך , װאָס עס טוט זיך. און דערזעהט דו עפיס אַזױנס , װאָס דן אױסטערליש שען און גוט , האַפּ עס און פלה ערמיט אַרױף ; עס װעט זײן אַ מתּנה פאַר די צריקים אַין גן-עדן . . . און מיט דער מתּנה אין דער האַנד זאָשטו אָנקאפּען און מעלדען זיך אָין מײן נאָמען בּײ'ם מלאָך פון דער פורטקע. זאָג : אָיך

15

.,און אַז דו װעט האָבּען געבּראַכט דרײ מתּנות , די זיכער — װעלען זיך פאַר דיר עפענען ד טױערן פון גן-עדן , זײ װעלן פועל'ן . . . בּײ'מ הכּטוק האָט מען ליעבּ גישט ד פײן געבּױרענע , נור ה אױסגעקומענע . . .

און אַזױ רײדענדיק , שטופּט ער זי מיט דהמנות פון הימעל ארױס .

\stopsection
\startsection[title={די ערשטע מתּנה}]

פליהט עס אַזױ דאָס דמע נשמה'לע , נידריג אַיבּער דער לעבּעדיגער װנ: , און זוכט מתּנות פאַר די צדיקים אין גן-עדן . און עס פליהט אַרום און אַרום , אַיבּער דערפער און שטעדט , װוּ נור אַ שטיקעל ישוכ ,— צװישען בּרענענדגע שטראַהלען אין די גרעסטע היצען ; אין רעגען-שיט— צװישען טראָפּען און װאַסער-נאָדלען ; סוף זומער — צרישען זילבּערנעם שפּינװעבּס , װאָס הענגט אין דער לופט , װינטער צװישען די שנײעלך ,. װאָס פאַלען פון דער הײך ... און עקט און קוקט , און קוקט דך ד אױגען אױס . . .

װוּ זי עדזעהט אַ יוד , פ*הט ד האַסטיג צו און קוקט אַיהק אַין די אױגען אַרין — צי געהט ער נישט מקדש השם זײן ?

ואו עס ליכט זיך בּײ נאַכט דודך אַ שפּאַלט פון אַ לאָדען , — זי איז דאָ און קוקט אַרײן , צי װאַקסען נישט אין דער שטישר שטום גאָט'ס שמעקענדיגע בּלמעליך — בּעהאַלטענע מעאָם טובים ל

לײדקנד ... מעהרסטענטײלס שפרינגט זי אָבּ פון אױגען און מענמער , דערשראָקען און פערציטערט . . .

און זען עס געהען װײטער פאָרבּײ זמנים און יאָהרען , פאַל זי מעט אין אַ מדה-טורה אַרײן . פון שטעדט זעגען שױן בּית-ע*מינ'ס געראָרען. בּית-מינ'ס האָט מען ױן שיף פעלער צואַקערט ; װעלדער האָט מען אױסגעהאַקט , פון שטײנער בּײ װאַסערן זזין זאַמד געװאָרען , טײכ'ן האָבּען ש געלגזנר געניטען , מײזענער שטערען זענען אַראָבּגעפאַלען , מילאָגע גש'לעך זענען אַרױבּעאיגען , — און זײן ליעבּער נאָמען האָט דך

סאָלקסטימליכע נעשיכטען .

אָן איהד ניש ערמאַנט , און קײן אױטערליש גוטס און שענס האָט זי נישט געמנען . . .

טראַכט זי בּײ זיך :

„די װעלט איז אַזױ אָרים , די מענען — אַזױ מיטעלמעסיג און נתי אױף דער נשמה , און די מעשס זײערע זענען אַעי קלין . . . װי קומט צו זײ „אַרסטעדלש" ז אױף אײבּיג בּין איך גע ונד און שרשטױסען ...

טר װי זי טראַכט אַזר , שלאָגט אַיהד אַ רױטע אם און ר אױגען דײן , אין מיטען דער פינסטערער , שוערער נאַכט אַ װיטע אם .

זי קוקט זיך אום — פון אַ הציך פענסטער ט די אַם . . .

גלים זענען אַ גכיר בּעפאַלען , גזלניס מיט מאַסקען אױף ד פנימ'ער. אײנער האַל אַ כּרענענדגען שטורקאַץ אַין דער האַנד און לײכט ; אַ צװײטער שטעהט בּײ'ם גביר'ס רוסט מיט אַ בּלצענדיק מעסער און הד'ט אַיהם א אַינאײנעם אַיבּער : „ריהרסט דו דיך , יוד , אַין דײן סוף ! גדט רד שפּיץ פון מעסער פון יענער זײט רוקען אַרױס" ! . . , און ד רעשט עפענט קאַסטען און שרענק און רױבּט .

און דער יוד שטעהט בּײ'ס מעסער און קוקט געאָסען צו . . . קײן רעם אַיבּער זײנע הדאָרע אױגען , קײן האָר פון דער װײסער בּאָרד בּין צו די לנדען — ריהרט זיך נישט . . . נישט זײן זאַך . . . „גאָט האָט געגעבּען , גאָט האָט גענומען , — טראַכט ער — געליבּט אין זײן הײליגער נשען ! „מיט דעס װערט מען נישט געבּױרען און דאָס גיט מען נישט מיט אין קנר אַרײן" , שעפּציען זײנע בּלאַסע ליפען .

און ער קוקט גאַנץ רוהיג צו , װי מען עפענט ד לעצטע יטוסלאַד פון דער לעצטער קשאָדע און װי מען עסט דױס זעקלעך מיט גאָלד און זילבער , זעקלעך מיט ציהדוג און אַלערלײ טײערע כּלים , — און ער שװײגט . . .

און אפשר אין ער זײ גאָר מפקיר !

טר פּלוצלונג , אַן די גזלנים טרעפען צום לעצטען בּעהעלטעניש און עיהען אַרױס אַ קלין זעקעלע , דאָס לעצטע , דאָס בּעהאַלטענסטע , — פערגעסט ער זיך , ציטערט ער אױף , צופאמען זיך איהם די אױגען , ציהש ער אױס צום װעהדען י רענטע האַר , װיל ער אַ געשרײ טהען :

„ריהרט נישט" !

נור אױפ'ן דט פון אַ געשרײ שפּרשט אַרױס אַ רױטער שטראַהל אָת דױנעררגען בּלוט — דאָס מעסער האָט זײנס געטון . . . עס שז

דאָס בּלוט פו ן האַר און שפּריצט אױפ'ן זעקעל אַרױף !

ער פאַג' און געשרינד רײסען הי גזל יס אױף דאָס זעקעל — דאָ װעט זײן דאָס בּעסטע , דאָס טײערסטע !

זײ האָסנען אָבּער אַ בּיטערן טעות געהאַט ; אומזיסט בּלוט מערגאָסען — נישע קײן זיל ער , נישט קײן גאָל אן נישט קײן ציהרונג אַין אין זעקעל געװען ; גאָר נישט מן דעם װאָס אָיז טײער און האָט אַ װערטה אױף ער דאָזיגער װעלט ! עס איז געװען אַ בּיסעל ערר , ארץ-ישראערה אין קכר אַרײן , און דאָס האָט דער גניר נעװאָא ראַטעװען פאַר פרעמדע הענד און אױגען און מיט דין בּלוט בּעשפּריצט . . . .

האַסט די נשמה אַ פערבּלוטיגטען שטױבּ פון ארץ-ישרא -ערד און מעע'רעט זיך דערהיט בּײ דער סורטקע אין הימעל . . .

די ערעיטע מנה אין אָעענומען געװאָרען .

\stopsection
\startsection[title={די צװײטע מתּנה}]

— גדזננק , האָ ער מפּך צוגערופען , פערמאַכענדיק הינטער איוע ח פורטקע פ.ין הימעל : נאָך צחי מתּנות ז

גאָ װעט העלפען ! האָסט ד גשמ און פרהש מעהל שרק אַראָבּ .

ד פר'יד האָט מיט דער צײט אָפּער אױפגעהערט . עס געהען װײמער אַרעק יאָהרען און יאָהרען , און ד זעהט נישט קײן אױסטערל שענס . . . און עס בּעפאַלען זי צוריק י טרױעריגע געדאַנקען :

„רי אַ לעבּעדיגער קװאַל האָ ח װעלט אַרױסגעשלאָגען פון גאָט'ס דילק און רנט און רינט װײטער אין ער צײט . און װאָר װײטער זי דיט , מדד ערה און שטױבּ נעהמט ד דן זיך דײן ; טריעבּער , אומרײנער װערט ד ; װעניגער מתּנות געפינט מען ליף אָיהד פאַר'ן על.... ק ענער װערזון ר מענען , דרשנער — די שװת , שטױבּיגער — ד ערות ; מעערם — נישט מט'ן אױג צו דערזעהן ! . . .

„רען ו.אָט — טראַכט ד ורטעד— זאָל הײסען איבּערװעגען מיט שאָל ר מעות מיט ד עברות פון דער האָנצער װעלט , װל זיך אוּיך דאָס

18

פּאָלקסטיטליכע געשיכטען .

צינעל קױם גערהרט , קױם-קױם געציטערט . . .

„ד װעלט קען ארך נישט אַראָבּ און נישט אַרױף . . . ד איז אױך נע ונד צװישען *כטיגען הימעל און פינסטערן שאול תּההיה . . . און דער אָיטר מיט'ן סניטר װלטען זיך דעריבּער אײבּיג געראַנגעלט , װי עסנעפּ נעפּ זיך דאָ אײבּיג ליכט מיט פינסטערקײט , װאַריטקײט מיט קעלט , לבּען מיט טױט . . .

„זי װיעגט זיך ד װעלט , און קען נישט — נישט אַרױף , נישט אַראָבּ , און עס װעלן דעריבּער אײבּיג זײן התונות און גט'ן , בּרית'ן און לוטת , סעודות-מצוה און סשדות-הדאה . . . און לועבּשסט און האַס . . . אַײבּיג , אײבּיג" . . .

פּל עלונג װעקען זיך קולות פון טראָמפּײטען און הערנער . . .

זי קוקט אַראָבּ — אַ דײטשע שטאָדט (פערשטעהט זיך , אַ פאַרצײטישע) , אַלערלױ , אױסגעמיגענע דענער רינגלען אַרום דעם פּלאַץ פאַר'ן מאַגיטדאַט , און פול מיט פערשיעדענ-פאַרבּיג-געק ייִדעטע מענשען אין דער פלאַץ ; פול קעפּ אַין די פענסטער ; מענען ליעגען אױף די דעכער אַרום , טײל זיצען רײטענדיק אױף ד בּאַלקען , װאָס יטטארען אַרױס בּײ ד װענד אונטער ד דעכער , פול געפראָפט זענען די בּאַלקאָנען . . .

פאַר'ן מאַגיסטראַט שטעהט אַ טיש , געדעקט מיט אַ גרין טוך מיט גלדענע אנזען און טראלבּען . בּײ'ס מיש זיצען די העררען פון מאַגיסטראַט , אין סשעטענע קלײדער אױף גלענע האָקלען , דן סױבּעלנע היטלען מיט װײסע פעערן אױף דיליאַנטענע קנעס ; אױבּען-ש זיצט דער פרעזידענט אַלין . אַ פרעסיגער אָדלער װעהט אַיהם איבּער-ן קאָפּ . . . אָן אַ זײט שטעהט געבּונדען אַ יודיש מײדעל . נישט װײט האַלטען צעהן קנעכט אַ װיל 'פערד . און דער פּרעזידענט הױבּט זיך אױף און יענט פאַר'ן יודישען מײדעל אַ פּק פון אַ פּאַפּיער . און ער װענדט זיך מיט'ן פּנים צום מאַרק

„אָט די ייִדין דאָ , ד יודישע מאָנטער , האָט אַ שװערע זינד בּעגאַנגק ; אַ שװערע זינד , װאָס גאָ אַלײן , װי גרױס זײן בּאַרמהערעיגקײט זאָל נישט זײן , װאָלט אַיהר נישט געקענט מוהל זײן . . .

,זי האָט זיך דױסגעגנכ'עט פון גהעטאַ און אין דומעגאַנגען , אין אונזער הײ*גען לעטען יום-טוב , אױף אונזערע רײנע גאַסען...

„און זי האָט בּעפלעקט מיט אַיהרע אונפערשעמטע אױגען אונשד הײלגע פּרשעסיע ; אונזערע הײליגע בּילדער, װאָס מיר האָרען מיט לױבּ—

19

י. ל. פרץ.

געזאַנג און קעסעל-פױקען אַיבּער ד גאַסען געטראָגען . . .

„מיט אַיהרע פערשאָלטענע אױערן האָט זי אײנגעזאַפט דאָס געשנג פון אונזערע װײס געקלײדטע , אונשולדיגע קינדער און דאָס קלאַנען פון די הײלגע פױקען . . . און װער װײט , ד האָט דער טײװעל, דער אוטרײנער טײרעל , װאָס האָט אױף זיך גענומען די געשטאַלט פון רער יודישעױ טאָכטער , פון פערשאָלטענעם רב'ס טאָכטער , דך נישט צוגעריהדט און נישט בּעפלעקט אונזער אַ דײניגקײט ?

„ראָס האָט ער געװאָלט , ער טײרעל אין דער שנער געשטאַ,טז ראָרוס ליקענעזי קען איך נישט — שען איז זי , שען , װי נור אַ טײװעל קען זיך מאַסנען , — זעהט די הוצפה-שטראַהלענדיגע אױגען פון אונטער די צנוד'דיג אַױאָבּגעלאָזטע זײדענע בּרעמען . . . זעהט דאָס אַלבּאַסטער פּנים , װאָס אַין אײן דער נגער תּפיסה גור בּלאַסער , אָבּער נישט טונקעלער געראָרען ! . . . זעהט איהרע פינגער , איהרע שמאָלע , לאַנגע פינגער בּײ ד הענד ; ױי זון לױכט דורך זײ אַדורך ! . . .

„אין זאָס האָט ער געװאָלט דער טײװעל , אָנרײסען אַ נשמה פון דבקות אָן דיער פראָצעסיע . . . און עס איז איהם געראָטען :

י זעו'ט דאָס שענע מײדעל ! האָט אױסגעו'ופען אונזערער אַ ריטטער פון אַײנער סון אונזערע שענסטע משפּערן . . .

און דאָס אַין שױן נערק אַיבּער דער מאָס — די האלעבּארדניקעס האָבּק זי בּעמערקט און געשפּט — ער האָט זיך אפילו ניט געװעהרט , ער טײרעל — װאָרום , רי אַזױ ? רײן זענען זײ דעמאָלט געװען , אָבּגע— רײנינט פון אַל זינד , האָט ער אױף זײ קײן טה נישט געהאַט . . .

און דאָס אָיז דעם טײװעל'ס , אין געשאַלט פון דעם יורי מײ— על , פּק :

„צובּינדען זאָל מען זי בּײ די דאָר , בּײ די לאַנגע טײװלאָנישע צעפּ , צום װײדעל פון אָט דעם װילען פערד . . .

„זאָל עס לױפען און עסען זי װי „אַ הורג" אַיבּער די גערען מאָס אַיהרע פיס האָבּען געטראָטען געגען אונזער הײליגען געזעץ . . .

„זאָל זגיהד בּלוט בּעשפּרצק און אָבּװאַשען ד שטײנער, װאָס זי האָט סעחומרײנינים מיט אַיהרע פיס" ! —

אַ װילר פרײד-געדײ האָט דך אױסגעריסע פֿון לא מײלר אַרום, ד אַן ד סאַליע פון ד לע געשרײען אַיז אַריבּער , פרעגט מען ד" פעדמשפּט'ע צום טױמ , אױבּ ד האָט דך ניא ערע דאָס צום לעטען

סצד'קסטימליכע נעשיכטען .

מאָל ש װינשען .

— אַיך האָבּ , ענטפערט ד געאָסען : עט*כע שפּילקעס בּעט איך !

— זי אין דול פאַר שדעק ! — מײנען די העררען פון מאַגיסטראַט .

— נײן ! ענטפערט ד רוהיג און קאל : דאָס אָין מײן געטער דיאן אין פעלאַע .

מען האָט עס איהד צולעבּ געטון . . .

— און אַצונד , קאָמאַנדירט דער פרעזידענט , בּינדט צו

עס געהען צו האלעבּארדניקעס און בּינדען צו מיט דטערדיגע הענד דעם רב'ס טאָכטער'ס שװאַרצע לאַעע צעפּ שם װיִידעל פון װילדען פערר, גאָס מען קען שײן קױם דערהאַלטען . . .

— אכט אַ װאַרע ! קאָמאַנד דט װײטער דער פּרעזידענט צו'ם עולס אױס'ן פלאַץ , און עס װערט אַ גערודער. דער עולם שטעלט זיך און דרינגט זיך עו צו ד װענד פון די הײזער , און אַלע הױבּען אױף די הענד , װער מיט אַ בּײטש , װער מיט אַ שפּיץ-רוט , װער מיט אַ טוך אין דער האַנד, אזן אַל זענ ן רײט צו יאָגען דש מילדע פערד , יעדערענס אָטהעם איז געשטיקט , אַלע פנימ'ער פלאַמען , אַלע אױגען בּליצען , און אַין * גערודער בּעמערקט נישט קײנער , װי די פער'משפּט'ע בּײגט זיך שטיל אַראָבּ און שפּילעט זיך צו די זױם פונ'ם קלײד צו די פיס און שטעקט טיעף, טיעף אַרײן די שפּילקעס זיך אין לײבּ — עס זאָל איהר לײבּ נישט יננטפלעקט װערען , װען דאָס פערד װעט זי שלעפען אין די גאַסען...

בּעמערקט האָט עס נאָר די גע ונד'ניצע — די נשמה...

— לאָזט דאָס פערד ! קאָמאַנדירט דערװײל װײטער דער פרעזידענט . זען ד קנעכט זענען דערפון אָפּגעשפּרונגען , און מיט אַ מאָל האָט עס זיך אױסגעריסען. און אױסגעריסען האָט זיך אױך אַ געשרײ פון אַלע מײלער , און עס װעהען און פײפען אין דער לוא אַלע בּײטשען , שפּיץדומען און טיכער. און װילד דערשראָקען יאָגט דאָס פערד איבּער'ן מאַרק , איבּער גאַסען און הינטער-גאַסען אַרױס , אַרױס פון שטאָדט . . .

און ד נשמה , די קע-ונד'נשע , האָט שױן אַרױעעצױגען אַ פער— בּלוטיגטע שפּילקע פון דער פער'משפּט'ערס סוס און פליהט שױן מיט איהד צום הימעל אַרױף !

— גאָך אײן מתּנה אינגאַנצען ! טרײסט ד דער מלאָך בּײ ער פורטקע.

\stopsection
\startsection[title={די דריטע מתּנה}]

און צזיריק הט אַראָבּ ד נשמה , אינגאַנצען נאָך אײן מתּנה דאַרף זי.

און עס געהען װײטער אַועק זמנים און יאָהרען , און עס בּעאלט ד װײטער אַ מרה דורה . ד װלט , האָכט זיך איהר , איז נאָך קלענער געװאָרען . . . נאָך קלענערע מעניעען , נאָך קלענערע מעשמ... די ניטע מד לעכשע . . .

אײנמסר מראַכט זי

„רען גאָט , געלױט זל זײן זײן נדען , זאָל װעלען אַ מאָל אָנשטעלן אוּין סוף כל סוף משפּט'ען ד װעלט , אַזױ רי עס אין , מיט אַמאָל... און פון אײן זײט זאָל זיך שטעלען אַ סניגור און שיטען פון װײסען זאַק אַרױס עטעלעך און שטױבּעלעך ; און פון דער צװײטער זײט— ער קטיגור זאָל שיטען זײנע פיצלעך און בּרעקלעך , ראָלט געדױערט און געדױערט אײדער די זעק װאָלטען לעדיק געװאָרען... אַזױ פיעל קלײגיגקײטען , אזױ פיעל

„אָבּער אַן די זעקלעך װאָלטען שײן לעדג געװאָרען , װאָס מאָלט גדק ז

„דאָס צינגעל װאָלט געװיס געבּליבּען שטעהן אין דער מיט !

„בּײ אַזױנע קלײניגקײטען , בּײ אַזױ פיעל קלײניגקײטען קען גאָר

נישט איבּען-װעגען... װאָ-ום װאָס? נאָך אַ פעערל , נאָך אַ שטרױעלע , נך אַ פּלעױוע , גאָך אַ שטױבּ ...

„און װאָס װאָלט גאָט געטון ז װאָס װלט ער גע'פּסק'עט ז

„צוריק צו תּהו גהו ז נישט : די עבירות װעגען נישט איבּער די מצװת..

„אױסלײזען ז אױך נישט : ד מעװת װעגען גישט איבּער די עבירות...

„װאָס דן ז

— געה װײטער ! װאָלט ד געזאָגט . . . ליה װײטער צװישען גיהנם ק גן-עדן , ליעבּשפט אין דאַס , דהמנות'דינע טרעהדען און רױכעריג בלוט... צװישן װיעגען און קנריס... רײטער , רײטער

דער נ מה אין אָנער בּעשערט געװען אױסגעלױזט צו װערען. פון מערע מהיעבות— װעקט זי אַ קול פון פױקען . . .

װוּ איז זי , װען ז

זי דערקענט נישט דעם אָרט , נישט די צײט...

פאָלקסטיטליכע נעשיכטען .

נד אַ פלאַץ פאַר אַ תּפיסה זעהט זי... אן די אײזערנע גרשען רן דק ינע פענאערלעך ציהען זיך שפּיעלענדיק די שטראַהלען שן ער זון... זײ גליטשן זיך אױך אָן די שטיקען פון צוזאַמעעעשטעלטען געװעהד בּײ ד װאַר. די סאָלדאַטען האָבּען אין האַנד אַרײן— ריטער בּעקומען . . .

שן צורי לאַטע שורות , מיט אַ שמאָלן דורכגאַנג אין דערמיט , האָט מע זײ אױסגעשטעל : „דורך אַ סטרױ" װעט מען טרײבּען...

װעמען ז

עפּעט אַ יול מוט אַ צעריסען העמ.ד אױס'ן מדערן ליבּ , מיט אַ יאַרמולקע אױפ'ן האַלבּ-געגאָלטען קאָפ. אָט , פיהרט מען איהם צו.

פאַר װאָס קומ עס איהם ? װער װײסט , אַ האָרצײטישע זאַך ! אפשר פאַר אַ געה , אשר פאַר אַ גזלה צי אַ רעיהת, און אפשר אַ בּלטל... עס אין דאָך גשוען סאַ צײטען....

און ד זלנער שמײכלען און טראַכטען : נאָך װאָס האָט מען אונן אַעי פיעל גענומען און אױסגעשטעלט ? ער װעט נישט אױסהאַלטען ר העלט

נאָר אָט שטופט מען איהס צװישען די שורות אַרײן , אָט געהט ער... און ער געהט גלײך און שטרױכעלט גישט און פאַלט ניש . . . ער לײזט שטיץ און האַלט זײ אױס...

אַ ציס-צדן אפט אָן דעמאָלט די סאָלדאַ ען ! ער געהט נאָך , ער געהט

און ד רטער פײפען אין דער לופט װי די רושת און האַפּען דום שי לײבּ װי ד לאַעען. און דאָס בּלוט פון מאָגערן גוף שפּריצט און שפּריצט , און הערט נישט אױף צפּ שפּריצען !

הו—האַ ! דאָ—האַ !

אין טיטען טרעט אַ סאָלראַט צו הױך און װאַרפט דעם פער'משא'ן אַראָבּ די יאַרמאָלקע פון קאָפּ. נאָך עטליכע טריט בּעמערקט ער עס , ער פער'משפט'ער... ער גיט זיך אַ ריהד , גליך ער האָט אַ ישוב , און איז זיך מישב , אין דרעהש זיך אום : ער װעט נישט געהן עלױ הראש , און ער געהט צוריק בּין צום אָרט , װוּ די יאַרמאָלקע לועגט , ער בּױגט זיך אײן און דעיט זי אױף , און דרעהט זיך צוריק אױס און גערט װײטער , רוהיג , רױט פערבּלוטיגט , גאָר מיט דער יאָימאָלקע אױפ'ן קאָפּ. אַזױ גדש ער , בּין ער פאַלט...

און אַ-ו ער איז געפעלע , אין צוגעלױגען ד נשמה און געהשפּט ד יאַרמלקקן , װאָס האָ אַזזי פיעל אומױמיגע ימיץ געקאָסט , אן אין אַרױף מיט איהר צו ער פורטקע שן הימל.

און די דרע מתּגה אין אױך אָעענומע גערדען !

גען ח צדקים האָבּק זיך פאַר איהד געמיהט : ד טױערן מן גן-ק האָ ן זיך רהח גאָך די דרײ מתּנות געעפענט

— מת שענע מתּנות , אױמעלש שנע... צו-ניץ קומען זײ נישט. גאַר גישט ל ממיל , אָבּעה למ — אױמערליש .

\stopsection

\stopchapter
\stoptext

\stopcomponent
