% Corrected by: Marc Trius
% Email: derpayatz@gmail.com
% Text source: https://ocr.yiddishbookcenter.org/
% Date retrieved: 5/24/2020

\doublehyphendemerits=20000
\startcomponent dray-matones
\project reader

\starttext
\startchapter[title={דרײַ מתּנות}] [
	author={י. ל. פּרץ}
]

\startsection[title={אויבען באַ דער וואג}]

אַ מאָל, מיט יאָר און דורות צוריק, אַין ערגעץ נפֿטר 
געוואָרן אַ ייִד.

מילא, אַ ייִד אַיז נפֿטר געוואָרן―אייביק לעבען קען 
מען נישט―טוט מען אים זײַן רעכט… בּרענגט מען 
אים צו קבֿורת ישׂראל…

נאָך סתימת הגולל, דער יתוֹם זאָגט קדיש―פֿליט די נשמה 
אַרויף, צום משפט, צום בית-דין של מעלה…

קומט זי אַרויף , הענט שוין האָרט פאַר'ן בּית-דין די וואָג 
, אוּיף וול ער מען וועגט די עברות און די מצוות…

קומט דעם כּר-מגן'ס סניגור , זײַן געוועזענער יצר טוב — 
און שטעלט זיך מיט אַ קלאָר ווייס זעקיל ווי שניי אין דער 
האַנד , בּיי דער וואָגשאָל מון דער רעכטער זייט…

קומט דעם -מנן'ס קטיגור — זײַן געוועזענער יצר הרע , דער 
גע- וועזענער מסית ומדה — און שטעלט זיך , מיט אַ קויטיג 
זעקיל אַין דער האַנד , בּיי דער ראָג-שאָל אויף דער 
לינקער דיט…

אַין ווייסען קלאָדען זעקיל זענען מצוות , אַין בּרודיק 
שאַרעען ע— קיל — עבירות . דט דער סניגור , פון 
שניי-ווייסען זעקיל אויף דער וואָגשל פון דער רעכטער זייט , 
מצוות — שמעקען זיי ווי פארפומעס און ליכטען ווי די יעטערנלעך 
אַין הימעל .

שיט דער קטיגור פון בּרודגען זעקיל אַרויס , אויף .דער 
וואָג-שאָל מן דער נקער האַנד , עבירות — זענען די , ניש 
פאַר אַייך געדאַכט , שוואַרץ ווי ק יל און אַ רה האָבּען 
זיי — סאַרע-ראָדנע פעך און מאָלע .

שטעהט אַזוי די אָרימע נשמה און קוקט , און גאַפט — זי 
האָ זיך גאַר ניט געריכט , עס זאָל זײַן אַזאַ הלוק 
צווישען „גוט" און „שלעט" ;

י. ל. סריל

דאָרט האָט זי גאַנץ אָפט בּיידע נישט דערקענט , איינעם 
פאַר'ן אַנדערן גענומען .

און די וואָר-שאָלן הויבּען זיך סאַמעליך , אַתיף און 
אַראָבּ , אַמאָל די , אַמאָל יענע… און דאָס 
צינגעל בּיי דער וואָגישאָל אַין דער ה. ך ציטערט , נעהמט 
זיך אויף אַ האָר לינקס , אויף אַ האָר רעכטס…

טר אויף אַ האָר… און נישט מיט אַ מאָל ! אַ 
פּראָסטער יוד געווען , אָהן גרויסען „להכעיס" , ר אויך 
אָהן „קדוש השם-'… אַזוי — קליינע מצוה'ליד , קליינע 
עצה'ליד , שרעטעליך , שטויבּעליך… מיילמאָל — קוים 
מיט'ן אויג צו דערזעהן .

און פודעסטרעגען , אַז דאָס ציעעל רוקט זיך אויף אַ האָר 
רעכטס , הערט זיך דן ד עולמות דעליונים אַ קלאַע פון 
שמהה , פון נהת ; רואָ זיך עס אָבּער , הס ושלום , לינקס 
, לאָגט דך אַרויס אַ זיפץ פון ענות , בּין ער דערגרייכט 
צום כּסא הכבוד .

און ר אכים שיטען פּשעליך , מיט כּונה. אַ שרויט נאָך אַ 
שרויט , 'אַ שטויבּ נאָך אַ שטובּ , ווי פּשוט'ע בּעל 
-בּתּים ליציטירען „אַתּה הראית" , אָ פּרוטה און נאָך 
אַ פרוטה…

טר אַ בּרונעם שעפּט זיך אייך אויס , ד זעקליך ווערען ליידיק 
.

— שויחך ? פרעגט דער בּית-דין- מש — אויך אַ מלאך צווישען 
אלע

דעד יעדר טונ , ווי דער שר הרע , דרעהען אויס די זעקליך 
קאַסויר

נישטאָ מעהר ! און דעמאָלט געהט דער שמש צו צום צינגעל 
זעהן , ור עס האָט זיך אָבּזנעשטעלט : דעכטס צי לינקס .

קוקט זנר און קוקט אַזוי , און זעהט אַזוינס , וואָס עס איז 
נאָך ניט געשעהען , זייד הימעל און ערד זענען בּעשפען 
געוואָרען…

— וואָס דויערט אַזוי לאַנג י פרעגט דער אנ-בּית-דין .

שטאַמזנלט דער שמש :

— גיך ! דאָס צינגעל שטעהט אין סאַמע-ראָדנע מיט !…

די ענ-ות האָבּען גיך געווארגען מיט די מצוות .

— אַקוראַט ז פרעגט מען נאָך אַ מאָל פון בּות-דין לן 
מעל'ס טיש . קוקט זור נאָך אַ מאָל דער מש און ענטפערט :

אוין- דער האָר !

קאָט דיאָס בּית-דין — מלה אַ ישוב און גיט , נאָכ'ן 
ישוב הדעת

אַרויס אַ פּסק כהאי לשנא :

סאָלקסטימליכע געשיכטען .

„היות ד שאים וועגען נישט איבּער די מעלס טובים — קומט 
דער נשמה קיין גיהנם נישט…

,און פון צווייטען צד — וועגען ד מצוות נייטט מעהר פאַר די 
ענרות לי— קען מען איהר נישט עפענען די מויערן פון גן-עדן 
.

,ל כן נע-ונד זל זי זײַן…

„זאָל זי אַרוממהען און דער מיט , צווישן הימעל און ע.ד — 
צין גאָט וועט זיך אָן דהר דערמאַנען און דערבּאַרמען זיך 
, און צודופק צו זיך מיט זײַן גנאָד" .

און דער שמש נעהמט די נשמה און סיהרט ד אַרויס .

יאָמערט נעבּיך די נ מה און קלאָגט אויף איהר מזל .

— וואָס וויינסטו אַזוי ז רעגט ער זי: וועסט נישט האָבּען קיין 
נתת און רייד פון גן-עדן , וועסט דו דך אָבּער אויך נישט 
האָבּען קיין צער און יסורים פון גיהנם , — קרט !

הי נשמה אָבּער אָזט זיך נישט טרייסטען :

. בּעסער די גרעסטע יסורים , זאָגט זי , איידער גאָר נישט 
! גאָר נישט אַין שדעקליך !

קריגט ער אויף אַיהר , דער בּית-דין-שמש , רשנות , גיט ער 
אָיהה אַיין עצה :

„פליה , זאָגט ער אַיהר , גשה'לע , אַראָבּ און האַלט 
זיך נידריג אַרום דער לעבּעדיגער וולט…

„אין הימל אַריין , זאָגט ער , קוק נישט… וואָרום 
וואָס וועסטו חו זעהן אין יענער זייט הימעל? נייערט שטערנליך 
! און דאָס זענען ליכטיגע נור קאַלטע רואים , זיי ווייסען 
פון קיין רהמנות נישט , זיי וועל זיך נישט מיהען פאַר דיר , 
זיי וואן גאָט אָן דיר נישט דערמאַנען…

„מיהען זיך האָר אַיין אָרימער פערראָגעלטער נשקך קענען 
נור די צדיקים פון גן-עדן... און זיי , הער נשמה'ל , — זיי 
האָבּען ליעבּ מתּנות.... מענע מתּנות… עס אַין מין 
— גיט ער צו בּיטער*ך — די טנע סון היינטיגע צדיקים

„ה זע , נשמה'ש — זאָגט ער ווייטער — נידריג אַרום דער 
לעבעדיגער וועלט און מעק זיך צו ווי עס לעבּט זיך , וואָס 
עס טוט זיך. און דערזעהט דו עפיס אַזוינס , וואָס דן 
אויסטערליש שען און גוט , האַפּ עס און פלה ערמיט אַרויף ; 
עס וועט זײַן אַ מתּנה פאַר די צריקים אַין גן-עדן… 
און מיט דער מתּנה אין דער האַנד זאָשטו אָנקאפּען און 
מעלדען זיך אָין מיין נאָמען בּיי'ם מלאָך פון דער פורטקע. 
זאָג : אָיך

15

.,און אַז דו וועט האָבּען געבּראַכט דריי מתּנות , די 
זיכער — וועלען זיך פאַר דיר עפענען ד טויערן פון גן-עדן , 
זיי וועלן פועל'ן… בּיי'מ הכּטוק האָט מען ליעבּ גישט 
ד פיין געבּוירענע , נור ה אויסגעקומענע…

און אַזוי ריידענדיק , שטופּט ער זי מיט דהמנות פון הימעל 
ארויס .

\stopsection
\startsection[title={די ערשטע מתּנה}]

פליהט עס אַזוי דאָס דמע נשמה'לע , נידריג אַיבּער דער 
לעבּעדיגער וונ: , און זוכט מתּנות פאַר די צדיקים אין 
גן-עדן . און עס פליהט אַרום און אַרום , אַיבּער דערפער 
און שטעדט , וווּ נור אַ שטיקעל ישוכ ,— צווישען 
בּרענענדגע שטראַהלען אין די גרעסטע היצען ; אין 
רעגען-שיט— צווישען טראָפּען און וואַסער-נאָדלען ; סוף 
זומער — צרישען זילבּערנעם שפּינוועבּס , וואָס הענגט אין 
דער לופט , ווינטער צווישען די שנייעלך ,. וואָס פאַלען פון 
דער הייך ... און עקט און קוקט , און קוקט דך ד אויגען אויס 
. . .

וווּ זי עדזעהט אַ יוד , פ*הט ד האַסטיג צו און קוקט 
אַיהק אַין די אויגען אַרין — צי געהט ער נישט מקדש השם 
זײַן ?

ואו עס ליכט זיך בּיי נאַכט דודך אַ שפּאַלט פון אַ 
לאָדען , — זי איז דאָ און קוקט אַריין , צי וואַקסען נישט 
אין דער שטישר שטום גאָט'ס שמעקענדיגע בּלמעליך — 
בּעהאַלטענע מעאָם טובים ל

ליידקנד ... מעהרסטענטיילס שפרינגט זי אָבּ פון אויגען און 
מענמער , דערשראָקען און פערציטערט…

און זען עס געהען ווייטער פאָרבּיי זמנים און יאָהרען , 
פאַל זי מעט אין אַ מדה-טורה אַריין . פון שטעדט זעגען 
שוין בּית-ע*מינ'ס געראָרען. בּית-מינ'ס האָט מען וין שיף 
פעלער צואַקערט ; וועלדער האָט מען אויסגעהאַקט , פון 
שטיינער בּיי וואַסערן זזין זאַמד געוואָרען , טייכ'ן האָבּען 
ש געלגזנר געניטען , מייזענער שטערען זענען 
אַראָבּגעפאַלען , מילאָגע גש'לעך זענען אַרויבּעאיגען , — 
און זײַן ליעבּער נאָמען האָט דך

סאָלקסטימליכע נעשיכטען .

אָן איהד ניש ערמאַנט , און קיין אויטערליש גוטס און שענס 
האָט זי נישט געמנען…

טראַכט זי בּיי זיך :

„די וועלט איז אַזוי אָרים , די מענען — אַזוי מיטעלמעסיג 
און נתי אויף דער נשמה , און די מעשס זייערע זענען אַעי 
קלין… ווי קומט צו זיי „אַרסטעדלש" ז אויף אייבּיג בּין 
איך גע ונד און שרשטויסען ...

טר ווי זי טראַכט אַזר , שלאָגט אַיהד אַ רויטע אם און ר 
אויגען דיין , אין מיטען דער פינסטערער , שוערער נאַכט אַ 
וויטע אם .

זי קוקט זיך אום — פון אַ הציך פענסטער ט די אַם…

גלים זענען אַ גכיר בּעפאַלען , גזלניס מיט מאַסקען אויף 
ד פנימ'ער. איינער האַל אַ כּרענענדגען שטורקאַץ אַין דער 
האַנד און לייכט ; אַ צווייטער שטעהט בּיי'ם גביר'ס רוסט מיט 
אַ בּלצענדיק מעסער און הד'ט אַיהם א אַינאיינעם אַיבּער 
: „ריהרסט דו דיך , יוד , אַין דיין סוף ! גדט רד שפּיץ 
פון מעסער פון יענער זייט רוקען אַרויס" ! . . , און ד 
רעשט עפענט קאַסטען און שרענק און רויבּט .

און דער יוד שטעהט בּיי'ס מעסער און קוקט געאָסען צו . . 
. קיין רעם אַיבּער זיינע הדאָרע אויגען , קיין האָר פון דער 
ווייסער בּאָרד בּין צו די לנדען — ריהרט זיך נישט… 
נישט זײַן זאַך… „גאָט האָט געגעבּען , גאָט האָט 
גענומען , — טראַכט ער — געליבּט אין זײַן הייליגער נשען ! 
„מיט דעס ווערט מען נישט געבּוירען און דאָס גיט מען נישט 
מיט אין קנר אַריין" , שעפּציען זיינע בּלאַסע ליפען .

און ער קוקט גאַנץ רוהיג צו , ווי מען עפענט ד לעצטע 
יטוסלאַד פון דער לעצטער קשאָדע און ווי מען עסט דויס 
זעקלעך מיט גאָלד און זילבער , זעקלעך מיט ציהדוג און 
אַלערליי טייערע כּלים , — און ער שווייגט…

און אפשר אין ער זיי גאָר מפקיר !

טר פּלוצלונג , אַן די גזלנים טרעפען צום לעצטען 
בּעהעלטעניש און עיהען אַרויס אַ קלין זעקעלע , דאָס 
לעצטע , דאָס בּעהאַלטענסטע , — פערגעסט ער זיך , ציטערט 
ער אויף , צופאמען זיך איהם די אויגען , ציהש ער אויס צום 
וועהדען י רענטע האַר , וויל ער אַ געשריי טהען :

„ריהרט נישט" !

נור אויפ'ן דט פון אַ געשריי שפּרשט אַרויס אַ רויטער 
שטראַהל אָת דוינעררגען בּלוט — דאָס מעסער האָט זיינס 
געטון… עס שז

דאָס בּלוט פו ן האַר און שפּריצט אויפ'ן זעקעל אַרויף !

ער פאַג' און געשרינד רייסען הי גזל יס אויף דאָס זעקעל — 
דאָ וועט זײַן דאָס בּעסטע , דאָס טייערסטע !

זיי האָסנען אָבּער אַ בּיטערן טעות געהאַט ; אומזיסט 
בּלוט מערגאָסען — נישע קיין זיל ער , נישט קיין גאָל אן 
נישט קיין ציהרונג אַין אין זעקעל געווען ; גאָר נישט מן 
דעם וואָס אָיז טייער און האָט אַ ווערטה אויף ער דאָזיגער 
וועלט ! עס איז געווען אַ בּיסעל ערר , ארץ-ישראערה אין 
קכר אַריין , און דאָס האָט דער גניר נעוואָא ראַטעווען 
פאַר פרעמדע הענד און אויגען און מיט דין בּלוט בּעשפּריצט 
.…

האַסט די נשמה אַ פערבּלוטיגטען שטויבּ פון ארץ-ישרא -ערד 
און מעע'רעט זיך דערהיט בּיי דער סורטקע אין הימעל…

די ערעיטע מנה אין אָעענומען געוואָרען .

\stopsection
\startsection[title={די צווייטע מתּנה}]

— גדזננק , האָ ער מפּך צוגערופען , פערמאַכענדיק הינטער 
איוע ח פורטקע פ.ין הימעל : נאָך צחי מתּנות ז

גאָ וועט העלפען ! האָסט ד גשמ און פרהש מעהל שרק אַראָבּ 
.

ד פר'יד האָט מיט דער צייט אָפּער אויפגעהערט . עס געהען 
וויימער אַרעק יאָהרען און יאָהרען , און ד זעהט נישט קיין 
אויסטערל שענס… און עס בּעפאַלען זי צוריק י 
טרויעריגע געדאַנקען :

„רי אַ לעבּעדיגער קוואַל האָ ח וועלט אַרויסגעשלאָגען פון 
גאָט'ס דילק און רנט און רינט ווייטער אין ער צייט . און 
וואָר ווייטער זי דיט , מדד ערה און שטויבּ נעהמט ד דן זיך 
דיין ; טריעבּער , אומריינער ווערט ד ; וועניגער מתּנות 
געפינט מען ליף אָיהד פאַר'ן על.... ק ענער ווערזון ר 
מענען , דרשנער — די שוות , שטויבּיגער — ד ערות ; מעערם — 
נישט מט'ן אויג צו דערזעהן !…

„רען ו.אָט — טראַכט ד ורטעד— זאָל הייסען איבּערוועגען 
מיט שאָל ר מעות מיט ד עברות פון דער האָנצער וועלט , וול 
זיך אוּיך דאָס

18

פּאָלקסטיטליכע געשיכטען .

צינעל קוים גערהרט , קוים-קוים געציטערט…

„ד וועלט קען ארך נישט אַראָבּ און נישט אַרויף… ד 
איז אויך נע ונד צווישען *כטיגען הימעל און פינסטערן שאול 
תּההיה… און דער אָיטר מיט'ן סניטר וולטען זיך 
דעריבּער אייבּיג געראַנגעלט , ווי עסנעפּ נעפּ זיך דאָ 
אייבּיג ליכט מיט פינסטערקייט , וואַריטקייט מיט קעלט , 
לבּען מיט טויט…

„זי וויעגט זיך ד וועלט , און קען נישט — נישט אַרויף , 
נישט אַראָבּ , און עס וועלן דעריבּער אייבּיג זײַן התונות 
און גט'ן , בּרית'ן און לוטת , סעודות-מצוה און 
סשדות-הדאה… און לועבּשסט און האַס… אַייבּיג , 
אייבּיג"…

פּל עלונג וועקען זיך קולות פון טראָמפּייטען און הערנער . 
. .

זי קוקט אַראָבּ — אַ דייטשע שטאָדט (פערשטעהט זיך , אַ 
פאַרצייטישע) , אַלערלוי , אויסגעמיגענע דענער רינגלען 
אַרום דעם פּלאַץ פאַר'ן מאַגיטדאַט , און פול מיט 
פערשיעדענ-פאַרבּיג-געק ייִדעטע מענשען אין דער פלאַץ ; 
פול קעפּ אַין די פענסטער ; מענען ליעגען אויף די דעכער 
אַרום , טייל זיצען רייטענדיק אויף ד בּאַלקען , וואָס 
יטטארען אַרויס בּיי ד ווענד אונטער ד דעכער , פול געפראָפט 
זענען די בּאַלקאָנען…

פאַר'ן מאַגיסטראַט שטעהט אַ טיש , געדעקט מיט אַ גרין 
טוך מיט גלדענע אנזען און טראלבּען . בּיי'ס מיש זיצען די 
העררען פון מאַגיסטראַט , אין סשעטענע קליידער אויף גלענע 
האָקלען , דן סויבּעלנע היטלען מיט ווייסע פעערן אויף 
דיליאַנטענע קנעס ; אויבּען-ש זיצט דער פרעזידענט אַלין . 
אַ פרעסיגער אָדלער וועהט אַיהם איבּער-ן קאָפּ… אָן 
אַ זייט שטעהט געבּונדען אַ יודיש מיידעל . נישט ווייט 
האַלטען צעהן קנעכט אַ וויל 'פערד . און דער פּרעזידענט 
הויבּט זיך אויף און יענט פאַר'ן יודישען מיידעל אַ פּק פון 
אַ פּאַפּיער . און ער ווענדט זיך מיט'ן פּנים צום מאַרק

„אָט די ייִדין דאָ , ד יודישע מאָנטער , האָט אַ שווערע 
זינד בּעגאַנגק ; אַ שווערע זינד , וואָס גאָ אַליין , ווי 
גרויס זײַן בּאַרמהערעיגקייט זאָל נישט זײַן , וואָלט אַיהר 
נישט געקענט מוהל זײַן…

,זי האָט זיך דויסגעגנכ'עט פון גהעטאַ און אין 
דומעגאַנגען , אין אונזער היי*גען לעטען יום-טוב , אויף 
אונזערע ריינע גאַסען...

„און זי האָט בּעפלעקט מיט אַיהרע אונפערשעמטע אויגען 
אונשד היילגע פּרשעסיע ; אונזערע הייליגע בּילדער, וואָס 
מיר האָרען מיט לויבּ—

19

י. ל. פרץ.

געזאַנג און קעסעל-פויקען אַיבּער ד גאַסען געטראָגען . . 
.

„מיט אַיהרע פערשאָלטענע אויערן האָט זי איינגעזאַפט דאָס 
געשנג פון אונזערע ווייס געקליידטע , אונשולדיגע קינדער און 
דאָס קלאַנען פון די היילגע פויקען… און ווער ווייט , ד 
האָט דער טייוועל, דער אוטריינער טיירעל , וואָס האָט אויף זיך 
גענומען די געשטאַלט פון רער יודישעוי טאָכטער , פון 
פערשאָלטענעם רב'ס טאָכטער , דך נישט צוגעריהדט און נישט 
בּעפלעקט אונזער אַ דייניגקייט ?

„ראָס האָט ער געוואָלט , ער טיירעל אין דער שנער 
געשטאַ,טז ראָרוס ליקענעזי קען איך נישט — שען איז זי , 
שען , ווי נור אַ טייוועל קען זיך מאַסנען , — זעהט די 
הוצפה-שטראַהלענדיגע אויגען פון אונטער די צנוד'דיג 
אַויאָבּגעלאָזטע זיידענע בּרעמען… זעהט דאָס 
אַלבּאַסטער פּנים , וואָס אַין איין דער נגער תּפיסה גור 
בּלאַסער , אָבּער נישט טונקעלער געראָרען !… זעהט 
איהרע פינגער , איהרע שמאָלע , לאַנגע פינגער בּיי ד הענד 
; ויי זון לויכט דורך זיי אַדורך !…

„אין זאָס האָט ער געוואָלט דער טייוועל , אָנרייסען אַ נשמה 
פון דבקות אָן דיער פראָצעסיע… און עס איז איהם 
געראָטען :

י זעו'ט דאָס שענע מיידעל ! האָט אויסגעו'ופען אונזערער אַ 
ריטטער פון אַיינער סון אונזערע שענסטע משפּערן…

און דאָס אַין שוין נערק אַיבּער דער מאָס — די 
האלעבּארדניקעס האָבּק זי בּעמערקט און געשפּט — ער האָט 
זיך אפילו ניט געוועהרט , ער טיירעל — וואָרום , רי אַזוי ? 
ריין זענען זיי דעמאָלט געווען , אָבּגע— ריינינט פון אַל 
זינד , האָט ער אויף זיי קיין טה נישט געהאַט…

און דאָס אָיז דעם טייוועל'ס , אין געשאַלט פון דעם יורי 
מיי— על , פּק :

„צובּינדען זאָל מען זי בּיי די דאָר , בּיי די לאַנגע 
טייוולאָנישע צעפּ , צום וויידעל פון אָט דעם ווילען פערד . . 
.

„זאָל עס לויפען און עסען זי ווי „אַ הורג" אַיבּער די 
גערען מאָס אַיהרע פיס האָבּען געטראָטען געגען אונזער 
הייליגען געזעץ…

„זאָל זגיהד בּלוט בּעשפּרצק און אָבּוואַשען ד שטיינער, 
וואָס זי האָט סעחומריינינים מיט אַיהרע פיס" ! —

אַ ווילר פרייד-געדיי האָט דך אויסגעריסע פֿון לא מיילר 
אַרום, ד אַן ד סאַליע פון ד לע געשרייען אַיז אַריבּער , 
פרעגט מען ד" פעדמשפּט'ע צום טוימ , אויבּ ד האָט דך ניא 
ערע דאָס צום לעטען

סצד'קסטימליכע נעשיכטען .

מאָל ש ווינשען .

— אַיך האָבּ , ענטפערט ד געאָסען : עט*כע שפּילקעס בּעט 
איך !

— זי אין דול פאַר שדעק ! — מיינען די העררען פון 
מאַגיסטראַט .

— ניין ! ענטפערט ד רוהיג און קאל : דאָס אָין מיין געטער 
דיאן אין פעלאַע .

מען האָט עס איהד צולעבּ געטון…

— און אַצונד , קאָמאַנדירט דער פרעזידענט , בּינדט צו

עס געהען צו האלעבּארדניקעס און בּינדען צו מיט דטערדיגע 
הענד דעם רב'ס טאָכטער'ס שוואַרצע לאַעע צעפּ שם וויִידעל 
פון ווילדען פערר, גאָס מען קען שיין קוים דערהאַלטען…

— אכט אַ וואַרע ! קאָמאַנד דט ווייטער דער פּרעזידענט צו'ם 
עולס אויס'ן פלאַץ , און עס ווערט אַ גערודער. דער עולם 
שטעלט זיך און דרינגט זיך עו צו ד ווענד פון די הייזער , 
און אַלע הויבּען אויף די הענד , ווער מיט אַ בּייטש , ווער 
מיט אַ שפּיץ-רוט , ווער מיט אַ טוך אין דער האַנד, אזן 
אַל זענ ן רייט צו יאָגען דש מילדע פערד , יעדערענס 
אָטהעם איז געשטיקט , אַלע פנימ'ער פלאַמען , אַלע אויגען 
בּליצען , און אַין * גערודער בּעמערקט נישט קיינער , ווי 
די פער'משפּט'ע בּייגט זיך שטיל אַראָבּ און שפּילעט זיך 
צו די זוים פונ'ם קלייד צו די פיס און שטעקט טיעף, טיעף 
אַריין די שפּילקעס זיך אין לייבּ — עס זאָל איהר לייבּ 
נישט יננטפלעקט ווערען , ווען דאָס פערד וועט זי שלעפען אין 
די גאַסען...

בּעמערקט האָט עס נאָר די גע ונד'ניצע — די נשמה...

— לאָזט דאָס פערד ! קאָמאַנדירט דערווייל ווייטער דער 
פרעזידענט . זען ד קנעכט זענען דערפון אָפּגעשפּרונגען , 
און מיט אַ מאָל האָט עס זיך אויסגעריסען. און אויסגעריסען 
האָט זיך אויך אַ געשריי פון אַלע מיילער , און עס וועהען 
און פייפען אין דער לוא אַלע בּייטשען , שפּיץדומען און 
טיכער. און ווילד דערשראָקען יאָגט דאָס פערד איבּער'ן 
מאַרק , איבּער גאַסען און הינטער-גאַסען אַרויס , אַרויס 
פון שטאָדט…

און ד נשמה , די קע-ונד'נשע , האָט שוין אַרויעעצויגען אַ 
פער— בּלוטיגטע שפּילקע פון דער פער'משפּט'ערס סוס און 
פליהט שוין מיט איהד צום הימעל אַרויף !

— גאָך איין מתּנה אינגאַנצען ! טרייסט ד דער מלאָך בּיי ער 
פורטקע.

\stopsection
\startsection[title={די דריטע מתּנה}]

און צזיריק הט אַראָבּ ד נשמה , אינגאַנצען נאָך איין 
מתּנה דאַרף זי.

און עס געהען ווייטער אַועק זמנים און יאָהרען , און עס 
בּעאלט ד ווייטער אַ מרה דורה . ד וולט , האָכט זיך איהר , 
איז נאָך קלענער געוואָרען… נאָך קלענערע מעניעען , 
נאָך קלענערע מעשמ... די ניטע מד לעכשע…

איינמסר מראַכט זי

„רען גאָט , געלויט זל זײַן זײַן נדען , זאָל וועלען אַ מאָל 
אָנשטעלן אוּין סוף כל סוף משפּט'ען ד וועלט , אַזוי רי עס 
אין , מיט אַמאָל... און פון איין זייט זאָל זיך שטעלען אַ 
סניגור און שיטען פון ווייסען זאַק אַרויס עטעלעך און 
שטויבּעלעך ; און פון דער צווייטער זייט— ער קטיגור זאָל 
שיטען זיינע פיצלעך און בּרעקלעך , ראָלט געדויערט און 
געדויערט איידער די זעק וואָלטען לעדיק געוואָרען... אַזוי 
פיעל קלייגיגקייטען , אזוי פיעל

„אָבּער אַן די זעקלעך וואָלטען שיין לעדג געוואָרען , וואָס 
מאָלט גדק ז

„דאָס צינגעל וואָלט געוויס געבּליבּען שטעהן אין דער מיט !

„בּיי אַזוינע קלייניגקייטען , בּיי אַזוי פיעל קלייניגקייטען 
קען גאָר

נישט איבּען-וועגען... וואָ-ום וואָס? נאָך אַ פעערל , נאָך 
אַ שטרויעלע , נך אַ פּלעויוע , גאָך אַ שטויבּ ...

„און וואָס וואָלט גאָט געטון ז וואָס וולט ער גע'פּסק'עט ז

„צוריק צו תּהו גהו ז נישט : די עבירות וועגען נישט 
איבּער די מצוות..

„אויסלייזען ז אויך נישט : ד מעוות וועגען גישט איבּער די 
עבירות...

„וואָס דן ז

— געה ווייטער ! וואָלט ד געזאָגט… ליה ווייטער צווישען 
גיהנם ק גן-עדן , ליעבּשפט אין דאַס , דהמנות'דינע 
טרעהדען און רויכעריג בלוט... צווישן וויעגען און קנריס... 
רייטער , רייטער

דער נ מה אין אָנער בּעשערט געווען אויסגעלויזט צו ווערען. 
פון מערע מהיעבות— וועקט זי אַ קול פון פויקען…

וווּ איז זי , ווען ז

זי דערקענט נישט דעם אָרט , נישט די צייט...

פאָלקסטיטליכע נעשיכטען .

נד אַ פלאַץ פאַר אַ תּפיסה זעהט זי... אן די אייזערנע 
גרשען רן דק ינע פענאערלעך ציהען זיך שפּיעלענדיק די 
שטראַהלען שן ער זון... זיי גליטשן זיך אויך אָן די שטיקען 
פון צוזאַמעעעשטעלטען געוועהד בּיי ד וואַר. די סאָלדאַטען 
האָבּען אין האַנד אַריין— ריטער בּעקומען…

שן צורי לאַטע שורות , מיט אַ שמאָלן דורכגאַנג אין 
דערמיט , האָט מע זיי אויסגעשטעל : „דורך אַ סטרוי" וועט מען 
טרייבּען...

וועמען ז

עפּעט אַ יול מוט אַ צעריסען העמ.ד אויס'ן מדערן ליבּ , 
מיט אַ יאַרמולקע אויפ'ן האַלבּ-געגאָלטען קאָפ. אָט , 
פיהרט מען איהם צו.

פאַר וואָס קומ עס איהם ? ווער ווייסט , אַ האָרצייטישע זאַך 
! אפשר פאַר אַ געה , אשר פאַר אַ גזלה צי אַ רעיהת, און 
אפשר אַ בּלטל... עס אין דאָך גשוען סאַ צייטען....

און ד זלנער שמייכלען און טראַכטען : נאָך וואָס האָט מען 
אונן אַעי פיעל גענומען און אויסגעשטעלט ? ער וועט נישט 
אויסהאַלטען ר העלט

נאָר אָט שטופט מען איהס צווישען די שורות אַריין , אָט 
געהט ער... און ער געהט גלייך און שטרויכעלט גישט און 
פאַלט ניש… ער לייזט שטיץ און האַלט זיי אויס...

אַ ציס-צדן אפט אָן דעמאָלט די סאָלדאַ ען ! ער געהט 
נאָך , ער געהט

און ד רטער פייפען אין דער לופט ווי די רושת און האַפּען 
דום שי לייבּ ווי ד לאַעען. און דאָס בּלוט פון מאָגערן 
גוף שפּריצט און שפּריצט , און הערט נישט אויף צפּ 
שפּריצען !

הו—האַ ! דאָ—האַ !

אין טיטען טרעט אַ סאָלראַט צו הויך און וואַרפט דעם 
פער'משא'ן אַראָבּ די יאַרמאָלקע פון קאָפּ. נאָך עטליכע 
טריט בּעמערקט ער עס , ער פער'משפט'ער... ער גיט זיך אַ 
ריהד , גליך ער האָט אַ ישוב , און איז זיך מישב , אין 
דרעהש זיך אום : ער וועט נישט געהן עלוי הראש , און ער 
געהט צוריק בּין צום אָרט , וווּ די יאַרמאָלקע לועגט , 
ער בּויגט זיך איין און דעיט זי אויף , און דרעהט זיך צוריק 
אויס און גערט ווייטער , רוהיג , רויט פערבּלוטיגט , גאָר מיט 
דער יאָימאָלקע אויפ'ן קאָפּ. אַזוי גדש ער , בּין ער 
פאַלט...

און אַ-ו ער איז געפעלע , אין צוגעלויגען ד נשמה און 
געהשפּט ד יאַרמלקקן , וואָס האָ אַזזי פיעל אומוימיגע 
ימיץ געקאָסט , אן אין אַרויף מיט איהר צו ער פורטקע שן 
הימל.

און די דרע מתּגה אין אויך אָעענומע גערדען !

גען ח צדקים האָבּק זיך פאַר איהד געמיהט : ד טויערן מן 
גן-ק האָ ן זיך רהח גאָך די דריי מתּנות געעפענט

— מת שענע מתּנות , אוימעלש שנע... צו-ניץ קומען זיי נישט. 
גאַר גישט ל ממיל , אָבּעה למ — אוימערליש .

\stopsection

\stopchapter
\stoptext

\stopcomponent
