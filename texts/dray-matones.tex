% Corrected by: Marc Trius
% Email: derpayatz@gmail.com
% Text source: https://ocr.yiddishbookcenter.org/
% Date retrieved: 5/24/2020

\doublehyphendemerits=20000
\startcomponent dray־matones
\project reader

\starttext
\startchapter[title={דרײַ מתּנות}] [
	author={י. ל. פּרץ}
]

\startsection[title={.אויבן בײַ דער וואָג}]

אַ מאָל, מיט יאָר און דורות צוריק, אין ערגעץ נפֿטר 
געוואָרן אַ ייִד.

מילא, אַ ייִד אַיז נפֿטר געוואָרן―אייביק לעבען קען 
מען נישט―טוט מען אים זײַן רעכט… ברענגט מען 
אים צו קבֿורת ישׂראל…

נאָך סתימת הגולל, דער יתוֹם זאָגט קדיש―פֿליט די נשמה 
אַרויף, צום משפט, צום בית־דין של מעלה…

קומט זי אַרויף, הענגט שוין האָרט פאַרן בית־דין די וואָג, 
אַף וועלכער מען וועגט די עברות און די מצוות…

קומט דעם בר־מנןס סניגור, זײַן געוועזענער יצר טוב―
און שטעלט זיך מיט אַ קלאָר ווײַס זעקל ווי שניי אין דער 
האַנט, בײַ דער וואָגשאָל פֿון דער רעכטער זײַט…

קומט דעם בר־מנןס קטיגור―זײַן געוועזענער יצר הרע, דער 
געוועזענער מסית ומדיח―און שטעלט זיך, מיט אַ קויטיק 
זעקל אַין דער האַנט, בײַ דער וואָגשאָל אַף דער 
לינקער זײַט…

אין ווײַסן קלאָרן זעקל זענען מצוות, אַין ברודיק 
שוואַרצן זעקל―עבירות. שיט דער סניגור, פֿון 
שניי־ווײַסן זעקיל אַף דער וואָגשאָל פֿון דער רעכטער זײַט, 
מצוות―שמעקען זיי ווי פּארפֿומעס און לויכטן ווי די שטערנדלעך 
אין הימל.

שיט דער קטיגור פֿון ברודגען זעקיל אַרויס , אַף .דער 
וואָג־שאָל מן דער נקער האַנד , עבירות―זענען די , ניש 
פאַר אַייך געדאַכט , שוואַרץ ווי ק יל און אַ רה האָבען 
זיי―סאַרע־ראָדנע פעך און מאָלע .

שטעהט אַזוי די אָרימע נשמה און קוקט , און גאַפט―זי 
האָ זיך גאַר ניט געריכט , עס זאָל זײַן אַזאַ הלוק 
צווישען „גוט" און „שלעט" ;

י. ל. סריל

דאָרט האָט זי גאַנץ אָפט ביידע נישט דערקענט , איינעם 
פאַר'ן אַנדערן גענומען .

און די וואָר־שאָלן הויבען זיך סאַמעליך , אַתיף און 
אַראָב , אַמאָל די , אַמאָל יענע… און דאָס 
צינגעל בײַ דער וואָגישאָל אַין דער ה. ך ציטערט , נעהמט 
זיך אַף אַ האָר לינקס , אַף אַ האָר רעכטס…

טר אַף אַ האָר… און נישט מיט אַ מאָל ! אַ 
פּראָסטער יוד געווען , אָהן גרויסען „להכעיס" , ר אויך 
אָהן „קדוש השם־'… אַזוי―קליינע מצוה'ליד , קליינע 
עצה'ליד , שרעטעליך , שטויבעליך… מיילמאָל―קוים 
מיט'ן אויג צו דערזעהן .

און פודעסטרעגען , אַז דאָס ציעעל רוקט זיך אַף אַ האָר 
רעכטס , הערט זיך דן ד עולמות דעליונים אַ קלאַע פֿון 
שמהה , פֿון נהת ; רואָ זיך עס אָבער , הס ושלום , לינקס 
, לאָגט דך אַרויס אַ זיפץ פֿון ענות , בין ער דערגרייכט 
צום כּסא הכבוד .

און ר אכים שיטען פּשעליך , מיט כּונה. אַ שרויט נאָך אַ 
שרויט , 'אַ שטויב נאָך אַ שטוב , ווי פּשוט'ע בעל 
־בתּים ליציטירען „אַתּה הראית" , אָ פּרוטה און נאָך 
אַ פרוטה…

טר אַ ברונעם שעפּט זיך אייך אויס , ד זעקליך ווערען ליידיק 
.

— שויחך ? פרעגט דער בית־דין־ מש―אויך אַ מלאך צווישען 
אלע

דעד יעדר טונ , ווי דער שר הרע , דרעהען אויס די זעקליך 
קאַסויר

נישטאָ מעהר ! און דעמאָלט געהט דער שמש צו צום צינגעל 
זעהן , ור עס האָט זיך אָבזנעשטעלט : דעכטס צי לינקס .

קוקט זנר און קוקט אַזוי , און זעהט אַזוינס , וואָס עס איז 
נאָך ניט געשעהען , זייד הימעל און ערד זענען בעשפען 
געוואָרען…

— וואָס דויערט אַזוי לאַנג י פרעגט דער אנ־בית־דין .

שטאַמזנלט דער שמש :

— גיך ! דאָס צינגעל שטעהט אין סאַמע־ראָדנע מיט !…

די ענ־ות האָבען גיך געווארגען מיט די מצוות .

— אַקוראַט ז פרעגט מען נאָך אַ מאָל פֿון בות־דין לן 
מעל'ס טיש . קוקט זור נאָך אַ מאָל דער מש און ענטפערט :

אוין־ דער האָר !

קאָט דיאָס בית־דין―מלה אַ ישוב און גיט , נאָכ'ן 
ישוב הדעת

אַרויס אַ פּסק כהאי לשנא :

סאָלקסטימליכע געשיכטען .

„היות ד שאים וועגען נישט איבער די מעלס טובים―קומט 
דער נשמה קיין גיהנם נישט…

,און פֿון צווייטען צד―וועגען ד מצוות נייטט מעהר פאַר די 
ענרות לי— קען מען איהר נישט עפענען די מויערן פֿון גן־עדן 
.

,ל כן נע־ונד זל זי זײַן…

„זאָל זי אַרוממהען און דער מיט , צווישן הימעל און ע.ד―
צין גאָט וועט זיך אָן דהר דערמאַנען און דערבאַרמען זיך 
, און צודופק צו זיך מיט זײַן גנאָד" .

און דער שמש נעהמט די נשמה און סיהרט ד אַרויס .

יאָמערט נעביך די נ מה און קלאָגט אַף איהר מזל .

— וואָס וויינסטו אַזוי ז רעגט ער זי: וועסט נישט האָבען קיין 
נתת און רייד פֿון גן־עדן , וועסט דו דך אָבער אויך נישט 
האָבען קיין צער און יסורים פֿון גיהנם ,―קרט !

הי נשמה אָבער אָזט זיך נישט טרייסטען :

. בעסער די גרעסטע יסורים , זאָגט זי , איידער גאָר נישט 
! גאָר נישט אַין שדעקליך !

קריגט ער אַף אַיהר , דער בית־דין־שמש , רשנות , גיט ער 
אָיהה אַיין עצה :

„פליה , זאָגט ער אַיהר , גשה'לע , אַראָב און האַלט 
זיך נידריג אַרום דער לעבעדיגער וולט…

„אין הימל אַריין , זאָגט ער , קוק נישט… וואָרום 
וואָס וועסטו חו זעהן אין יענער זייט הימעל? נייערט שטערנליך 
! און דאָס זענען ליכטיגע נור קאַלטע רואים , זיי ווייסען 
פֿון קיין רהמנות נישט , זיי וועל זיך נישט מיהען פאַר דיר , 
זיי וואן גאָט אָן דיר נישט דערמאַנען…

„מיהען זיך האָר אַיין אָרימער פערראָגעלטער נשקך קענען 
נור די צדיקים פֿון גן־עדן... און זיי , הער נשמה'ל ,―זיי 
האָבען ליעב מתּנות.... מענע מתּנות… עס אַין מין 
— גיט ער צו ביטער*ך―די טנע סון היינטיגע צדיקים

„ה זע , נשמה'ש―זאָגט ער ווייטער―נידריג אַרום דער 
לעבעדיגער וועלט און מעק זיך צו ווי עס לעבט זיך , וואָס 
עס טוט זיך. און דערזעהט דו עפיס אַזוינס , וואָס דן 
אויסטערליש שען און גוט , האַפּ עס און פלה ערמיט אַרויף ; 
עס וועט זײַן אַ מתּנה פאַר די צריקים אַין גן־עדן… 
און מיט דער מתּנה אין דער האַנד זאָשטו אָנקאפּען און 
מעלדען זיך אָין מיין נאָמען בײַ'ם מלאָך פֿון דער פורטקע. 
זאָג : אָיך

15

.,און אַז דו וועט האָבען געבראַכט דריי מתּנות , די 
זיכער―וועלען זיך פאַר דיר עפענען ד טויערן פֿון גן־עדן , 
זיי וועלן פועל'ן… בײַ'מ הכּטוק האָט מען ליעב גישט 
ד פיין געבוירענע , נור ה אויסגעקומענע…

און אַזוי ריידענדיק , שטופּט ער זי מיט דהמנות פֿון הימעל 
ארויס .

\stopsection
\startsection[title={די ערשטע מתּנה}]

פליהט עס אַזוי דאָס דמע נשמה'לע , נידריג אַיבער דער 
לעבעדיגער וונ: , און זוכט מתּנות פאַר די צדיקים אין 
גן־עדן . און עס פליהט אַרום און אַרום , אַיבער דערפער 
און שטעדט , וווּ נור אַ שטיקעל ישוכ ,— צווישען 
ברענענדגע שטראַהלען אין די גרעסטע היצען ; אין 
רעגען־שיט— צווישען טראָפּען און וואַסער־נאָדלען ; סוף 
זומער―צרישען זילבערנעם שפּינוועבס , וואָס הענגט אין 
דער לופט , ווינטער צווישען די שנייעלך ,. וואָס פאַלען פֿון 
דער הייך ... און עקט און קוקט , און קוקט דך ד אויגען אויס 
. . .

וווּ זי עדזעהט אַ יוד , פ*הט ד האַסטיג צו און קוקט 
אַיהק אַין די אויגען אַרין―צי געהט ער נישט מקדש השם 
זײַן ?

ואו עס ליכט זיך בײַ נאַכט דודך אַ שפּאַלט פֿון אַ 
לאָדען ,―זי איז דאָ און קוקט אַריין , צי וואַקסען נישט 
אין דער שטישר שטום גאָט'ס שמעקענדיגע בלמעליך―
בעהאַלטענע מעאָם טובים ל

ליידקנד ... מעהרסטענטיילס שפרינגט זי אָב פֿון אויגען און 
מענמער , דערשראָקען און פערציטערט…

און זען עס געהען ווייטער פאָרביי זמנים און יאָהרען , 
פאַל זי מעט אין אַ מדה־טורה אַריין . פֿון שטעדט זעגען 
שוין בית־ע*מינ'ס געראָרען. בית־מינ'ס האָט מען וין שיף 
פעלער צואַקערט ; וועלדער האָט מען אויסגעהאַקט , פֿון 
שטיינער בײַ וואַסערן זזין זאַמד געוואָרען , טייכ'ן האָבען 
ש געלגזנר געניטען , מייזענער שטערען זענען 
אַראָבגעפאַלען , מילאָגע גש'לעך זענען אַרויבעאיגען ,―
און זײַן ליעבער נאָמען האָט דך

סאָלקסטימליכע נעשיכטען .

אָן איהד ניש ערמאַנט , און קיין אויטערליש גוטס און שענס 
האָט זי נישט געמנען…

טראַכט זי בײַ זיך :

„די וועלט איז אַזוי אָרים , די מענען―אַזוי מיטעלמעסיג 
און נתי אַף דער נשמה , און די מעשס זייערע זענען אַעי 
קלין… ווי קומט צו זיי „אַרסטעדלש" ז אַף אייביג בין 
איך גע ונד און שרשטויסען ...

טר ווי זי טראַכט אַזר , שלאָגט אַיהד אַ רויטע אם און ר 
אויגען דיין , אין מיטען דער פינסטערער , שוערער נאַכט אַ 
וויטע אם .

זי קוקט זיך אום―פֿון אַ הציך פענסטער ט די אַם…

גלים זענען אַ גכיר בעפאַלען , גזלניס מיט מאַסקען אַף 
ד פנימ'ער. איינער האַל אַ כּרענענדגען שטורקאַץ אַין דער 
האַנד און לייכט ; אַ צווייטער שטעהט בײַ'ם גביר'ס רוסט מיט 
אַ בלצענדיק מעסער און הד'ט אַיהם א אַינאיינעם אַיבער 
: „ריהרסט דו דיך , יוד , אַין דיין סוף ! גדט רד שפּיץ 
פֿון מעסער פֿון יענער זייט רוקען אַרויס" ! . . , און ד 
רעשט עפענט קאַסטען און שרענק און רויבט .

און דער יוד שטעהט בײַ'ס מעסער און קוקט געאָסען צו . . 
. קיין רעם אַיבער זיינע הדאָרע אויגען , קיין האָר פֿון דער 
ווייסער באָרד בין צו די לנדען―ריהרט זיך נישט… 
נישט זײַן זאַך… „גאָט האָט געגעבען , גאָט האָט 
גענומען ,―טראַכט ער―געליבט אין זײַן הייליגער נשען ! 
„מיט דעס ווערט מען נישט געבוירען און דאָס גיט מען נישט 
מיט אין קנר אַריין" , שעפּציען זיינע בלאַסע ליפען .

און ער קוקט גאַנץ רוהיג צו , ווי מען עפענט ד לעצטע 
יטוסלאַד פֿון דער לעצטער קשאָדע און ווי מען עסט דויס 
זעקלעך מיט גאָלד און זילבער , זעקלעך מיט ציהדוג און 
אַלערליי טייערע כּלים ,―און ער שווייגט…

און אפשר אין ער זיי גאָר מפקיר !

טר פּלוצלונג , אַן די גזלנים טרעפען צום לעצטען 
בעהעלטעניש און עיהען אַרויס אַ קלין זעקעלע , דאָס 
לעצטע , דאָס בעהאַלטענסטע ,―פערגעסט ער זיך , ציטערט 
ער אַף , צופאמען זיך איהם די אויגען , ציהש ער אויס צום 
וועהדען י רענטע האַר , וויל ער אַ געשריי טהען :

„ריהרט נישט" !

נור אויפ'ן דט פֿון אַ געשריי שפּרשט אַרויס אַ רויטער 
שטראַהל אָת דוינעררגען בלוט―דאָס מעסער האָט זיינס 
געטון… עס שז

דאָס בלוט פו ן האַר און שפּריצט אויפ'ן זעקעל אַרויף !

ער פאַג' און געשרינד רייסען הי גזל יס אַף דאָס זעקעל―
דאָ וועט זײַן דאָס בעסטע , דאָס טייערסטע !

זיי האָסנען אָבער אַ ביטערן טעות געהאַט ; אומזיסט 
בלוט מערגאָסען―נישע קיין זיל ער , נישט קיין גאָל אן 
נישט קיין ציהרונג אַין אין זעקעל געווען ; גאָר נישט מן 
דעם וואָס אָיז טייער און האָט אַ ווערטה אַף ער דאָזיגער 
וועלט ! עס איז געווען אַ ביסעל ערר , ארץ־ישראערה אין 
קכר אַריין , און דאָס האָט דער גניר נעוואָא ראַטעווען 
פאַר פרעמדע הענד און אויגען און מיט דין בלוט בעשפּריצט 
.…

האַסט די נשמה אַ פערבלוטיגטען שטויב פֿון ארץ־ישרא ־ערד 
און מעע'רעט זיך דערהיט בײַ דער סורטקע אין הימעל…

די ערעיטע מנה אין אָעענומען געוואָרען .

\stopsection
\startsection[title={די צווייטע מתּנה}]

— גדזננק , האָ ער מפּך צוגערופען , פערמאַכענדיק הינטער 
איוע ח פורטקע פ.ין הימעל : נאָך צחי מתּנות ז

גאָ וועט העלפען ! האָסט ד גשמ און פרהש מעהל שרק אַראָב 
.

ד פר'יד האָט מיט דער צייט אָפּער אויפגעהערט . עס געהען 
וויימער אַרעק יאָהרען און יאָהרען , און ד זעהט נישט קיין 
אויסטערל שענס… און עס בעפאַלען זי צוריק י 
טרויעריגע געדאַנקען :

„רי אַ לעבעדיגער קוואַל האָ ח וועלט אַרויסגעשלאָגען פֿון 
גאָט'ס דילק און רנט און רינט ווייטער אין ער צייט . און 
וואָר ווייטער זי דיט , מדד ערה און שטויב נעהמט ד דן זיך 
דיין ; טריעבער , אומריינער ווערט ד ; וועניגער מתּנות 
געפינט מען ליף אָיהד פאַר'ן על.... ק ענער ווערזון ר 
מענען , דרשנער―די שוות , שטויביגער―ד ערות ; מעערם―
נישט מט'ן אויג צו דערזעהן !…

„רען ו.אָט―טראַכט ד ורטעד— זאָל הייסען איבערוועגען 
מיט שאָל ר מעות מיט ד עברות פֿון דער האָנצער וועלט , וול 
זיך אוּיך דאָס

18

פּאָלקסטיטליכע געשיכטען .

צינעל קוים גערהרט , קוים־קוים געציטערט…

„ד וועלט קען ארך נישט אַראָב און נישט אַרויף… ד 
איז אויך נע ונד צווישען *כטיגען הימעל און פינסטערן שאול 
תּההיה… און דער אָיטר מיט'ן סניטר וולטען זיך 
דעריבער אייביג געראַנגעלט , ווי עסנעפּ נעפּ זיך דאָ 
אייביג ליכט מיט פינסטערקייט , וואַריטקייט מיט קעלט , 
לבען מיט טויט…

„זי וויעגט זיך ד וועלט , און קען נישט―נישט אַרויף , 
נישט אַראָב , און עס וועלן דעריבער אייביג זײַן התונות 
און גט'ן , ברית'ן און לוטת , סעודות־מצוה און 
סשדות־הדאה… און לועבשסט און האַס… אַייביג , 
אייביג"…

פּל עלונג וועקען זיך קולות פֿון טראָמפּייטען און הערנער . 
. .

זי קוקט אַראָב―אַ דייטשע שטאָדט (פערשטעהט זיך , אַ 
פאַרצייטישע) , אַלערלוי , אויסגעמיגענע דענער רינגלען 
אַרום דעם פּלאַץ פאַר'ן מאַגיטדאַט , און פול מיט 
פערשיעדענ־פאַרביג־געק ייִדעטע מענשען אין דער פלאַץ ; 
פול קעפּ אַין די פענסטער ; מענען ליעגען אַף די דעכער 
אַרום , טייל זיצען רייטענדיק אַף ד באַלקען , וואָס 
יטטארען אַרויס בײַ ד ווענד אונטער ד דעכער , פול געפראָפט 
זענען די באַלקאָנען…

פאַר'ן מאַגיסטראַט שטעהט אַ טיש , געדעקט מיט אַ גרין 
טוך מיט גלדענע אנזען און טראלבען . בײַ'ס מיש זיצען די 
העררען פֿון מאַגיסטראַט , אין סשעטענע קליידער אַף גלענע 
האָקלען , דן סויבעלנע היטלען מיט ווייסע פעערן אַף 
דיליאַנטענע קנעס ; אויבען־ש זיצט דער פרעזידענט אַלין . 
אַ פרעסיגער אָדלער וועהט אַיהם איבער־ן קאָפּ… אָן 
אַ זייט שטעהט געבונדען אַ יודיש מיידעל . נישט ווייט 
האַלטען צעהן קנעכט אַ וויל 'פערד . און דער פּרעזידענט 
הויבט זיך אַף און יענט פאַר'ן יודישען מיידעל אַ פּק פֿון 
אַ פּאַפּיער . און ער ווענדט זיך מיט'ן פּנים צום מאַרק

„אָט די ייִדין דאָ , ד יודישע מאָנטער , האָט אַ שווערע 
זינד בעגאַנגק ; אַ שווערע זינד , וואָס גאָ אַליין , ווי 
גרויס זײַן באַרמהערעיגקייט זאָל נישט זײַן , וואָלט אַיהר 
נישט געקענט מוהל זײַן…

,זי האָט זיך דויסגעגנכ'עט פֿון גהעטאַ און אין 
דומעגאַנגען , אין אונזער היי*גען לעטען יום־טוב , אַף 
אונזערע ריינע גאַסען...

„און זי האָט בעפלעקט מיט אַיהרע אונפערשעמטע אויגען 
אונשד היילגע פּרשעסיע ; אונזערע הייליגע בילדער, וואָס 
מיר האָרען מיט לויב—

19

י. ל. פרץ.

געזאַנג און קעסעל־פויקען אַיבער ד גאַסען געטראָגען . . 
.

„מיט אַיהרע פערשאָלטענע אויערן האָט זי איינגעזאַפט דאָס 
געשנג פֿון אונזערע ווייס געקליידטע , אונשולדיגע קינדער און 
דאָס קלאַנען פֿון די היילגע פויקען… און ווער ווייט , ד 
האָט דער טייוועל, דער אוטריינער טיירעל , וואָס האָט אַף זיך 
גענומען די געשטאַלט פֿון רער יודישעוי טאָכטער , פֿון 
פערשאָלטענעם רב'ס טאָכטער , דך נישט צוגעריהדט און נישט 
בעפלעקט אונזער אַ דייניגקייט ?

„ראָס האָט ער געוואָלט , ער טיירעל אין דער שנער 
געשטאַ,טז ראָרוס ליקענעזי קען איך נישט―שען איז זי , 
שען , ווי נור אַ טייוועל קען זיך מאַסנען ,―זעהט די 
הוצפה־שטראַהלענדיגע אויגען פֿון אונטער די צנוד'דיג 
אַויאָבגעלאָזטע זיידענע ברעמען… זעהט דאָס 
אַלבאַסטער פּנים , וואָס אַין איין דער נגער תּפיסה גור 
בלאַסער , אָבער נישט טונקעלער געראָרען !… זעהט 
איהרע פינגער , איהרע שמאָלע , לאַנגע פינגער בײַ ד הענד 
; ויי זון לויכט דורך זיי אַדורך !…

„אין זאָס האָט ער געוואָלט דער טייוועל , אָנרייסען אַ נשמה 
פֿון דבקות אָן דיער פראָצעסיע… און עס איז איהם 
געראָטען :

י זעו'ט דאָס שענע מיידעל ! האָט אויסגעו'ופען אונזערער אַ 
ריטטער פֿון אַיינער סון אונזערע שענסטע משפּערן…

און דאָס אַין שוין נערק אַיבער דער מאָס―די 
האלעבארדניקעס האָבק זי בעמערקט און געשפּט―ער האָט 
זיך אפילו ניט געוועהרט , ער טיירעל―וואָרום , רי אַזוי ? 
ריין זענען זיי דעמאָלט געווען , אָבגע— ריינינט פֿון אַל 
זינד , האָט ער אַף זיי קיין טה נישט געהאַט…

און דאָס אָיז דעם טייוועל'ס , אין געשאַלט פֿון דעם יורי 
מיי— על , פּק :

„צובינדען זאָל מען זי בײַ די דאָר , בײַ די לאַנגע 
טייוולאָנישע צעפּ , צום וויידעל פֿון אָט דעם ווילען פערד . . 
.

„זאָל עס לויפען און עסען זי ווי „אַ הורג" אַיבער די 
גערען מאָס אַיהרע פיס האָבען געטראָטען געגען אונזער 
הייליגען געזעץ…

„זאָל זגיהד בלוט בעשפּרצק און אָבוואַשען ד שטיינער, 
וואָס זי האָט סעחומריינינים מיט אַיהרע פיס" ! —

אַ ווילר פרייד־געדיי האָט דך אויסגעריסע פֿון לא מיילר 
אַרום, ד אַן ד סאַליע פֿון ד לע געשרייען אַיז אַריבער , 
פרעגט מען ד" פעדמשפּט'ע צום טוימ , אויב ד האָט דך ניא 
ערע דאָס צום לעטען

סצד'קסטימליכע נעשיכטען .

מאָל ש ווינשען .

— אַיך האָב , ענטפערט ד געאָסען : עט*כע שפּילקעס בעט 
איך !

— זי אין דול פאַר שדעק !―מיינען די העררען פֿון 
מאַגיסטראַט .

— ניין ! ענטפערט ד רוהיג און קאל : דאָס אָין מיין געטער 
דיאן אין פעלאַע .

מען האָט עס איהד צולעב געטון…

— און אַצונד , קאָמאַנדירט דער פרעזידענט , בינדט צו

עס געהען צו האלעבארדניקעס און בינדען צו מיט דטערדיגע 
הענד דעם רב'ס טאָכטער'ס שוואַרצע לאַעע צעפּ שם וויִידעל 
פֿון ווילדען פערר, גאָס מען קען שיין קוים דערהאַלטען…

— אכט אַ וואַרע ! קאָמאַנד דט ווייטער דער פּרעזידענט צו'ם 
עולס אויס'ן פלאַץ , און עס ווערט אַ גערודער. דער עולם 
שטעלט זיך און דרינגט זיך עו צו ד ווענד פֿון די הייזער , 
און אַלע הויבען אַף די הענד , ווער מיט אַ בייטש , ווער 
מיט אַ שפּיץ־רוט , ווער מיט אַ טוך אין דער האַנד, אזן 
אַל זענ ן רייט צו יאָגען דש מילדע פערד , יעדערענס 
אָטהעם איז געשטיקט , אַלע פנימ'ער פלאַמען , אַלע אויגען 
בליצען , און אַין * גערודער בעמערקט נישט קיינער , ווי 
די פער'משפּט'ע בייגט זיך שטיל אַראָב און שפּילעט זיך 
צו די זוים פונ'ם קלייד צו די פיס און שטעקט טיעף, טיעף 
אַריין די שפּילקעס זיך אין לייב―עס זאָל איהר לייב 
נישט יננטפלעקט ווערען , ווען דאָס פערד וועט זי שלעפען אין 
די גאַסען...

בעמערקט האָט עס נאָר די גע ונד'ניצע―די נשמה...

— לאָזט דאָס פערד ! קאָמאַנדירט דערווייל ווייטער דער 
פרעזידענט . זען ד קנעכט זענען דערפון אָפּגעשפּרונגען , 
און מיט אַ מאָל האָט עס זיך אויסגעריסען. און אויסגעריסען 
האָט זיך אויך אַ געשריי פֿון אַלע מיילער , און עס וועהען 
און פייפען אין דער לוא אַלע בייטשען , שפּיץדומען און 
טיכער. און ווילד דערשראָקען יאָגט דאָס פערד איבער'ן 
מאַרק , איבער גאַסען און הינטער־גאַסען אַרויס , אַרויס 
פֿון שטאָדט…

און ד נשמה , די קע־ונד'נשע , האָט שוין אַרויעעצויגען אַ 
פער— בלוטיגטע שפּילקע פֿון דער פער'משפּט'ערס סוס און 
פליהט שוין מיט איהד צום הימעל אַרויף !

— גאָך איין מתּנה אינגאַנצען ! טרייסט ד דער מלאָך בײַ ער 
פורטקע.

\stopsection
\startsection[title={די דריטע מתּנה}]

און צזיריק הט אַראָב ד נשמה , אינגאַנצען נאָך איין 
מתּנה דאַרף זי.

און עס געהען ווייטער אַועק זמנים און יאָהרען , און עס 
בעאלט ד ווייטער אַ מרה דורה . ד וולט , האָכט זיך איהר , 
איז נאָך קלענער געוואָרען… נאָך קלענערע מעניעען , 
נאָך קלענערע מעשמ... די ניטע מד לעכשע…

איינמסר מראַכט זי

„רען גאָט , געלויט זל זײַן זײַן נדען , זאָל וועלען אַ מאָל 
אָנשטעלן אוּין סוף כל סוף משפּט'ען ד וועלט , אַזוי רי עס 
אין , מיט אַמאָל... און פֿון איין זייט זאָל זיך שטעלען אַ 
סניגור און שיטען פֿון ווייסען זאַק אַרויס עטעלעך און 
שטויבעלעך ; און פֿון דער צווייטער זייט— ער קטיגור זאָל 
שיטען זיינע פיצלעך און ברעקלעך , ראָלט געדויערט און 
געדויערט איידער די זעק וואָלטען לעדיק געוואָרען... אַזוי 
פיעל קלייגיגקייטען , אזוי פיעל

„אָבער אַן די זעקלעך וואָלטען שיין לעדג געוואָרען , וואָס 
מאָלט גדק ז

„דאָס צינגעל וואָלט געוויס געבליבען שטעהן אין דער מיט !

„בײַ אַזוינע קלייניגקייטען , בײַ אַזוי פיעל קלייניגקייטען 
קען גאָר

נישט איבען־וועגען... וואָ־ום וואָס? נאָך אַ פעערל , נאָך 
אַ שטרויעלע , נך אַ פּלעויוע , גאָך אַ שטויב ...

„און וואָס וואָלט גאָט געטון ז וואָס וולט ער גע'פּסק'עט ז

„צוריק צו תּהו גהו ז נישט : די עבירות וועגען נישט 
איבער די מצוות..

„אויסלייזען ז אויך נישט : ד מעוות וועגען גישט איבער די 
עבירות...

„וואָס דן ז

— געה ווייטער ! וואָלט ד געזאָגט… ליה ווייטער צווישען 
גיהנם ק גן־עדן , ליעבשפט אין דאַס , דהמנות'דינע 
טרעהדען און רויכעריג בלוט... צווישן וויעגען און קנריס... 
רייטער , רייטער

דער נ מה אין אָנער בעשערט געווען אויסגעלויזט צו ווערען. 
פֿון מערע מהיעבות— וועקט זי אַ קול פֿון פויקען…

וווּ איז זי , ווען ז

זי דערקענט נישט דעם אָרט , נישט די צייט...

פאָלקסטיטליכע נעשיכטען .

נד אַ פלאַץ פאַר אַ תּפיסה זעהט זי... אן די אייזערנע 
גרשען רן דק ינע פענאערלעך ציהען זיך שפּיעלענדיק די 
שטראַהלען שן ער זון... זיי גליטשן זיך אויך אָן די שטיקען 
פֿון צוזאַמעעעשטעלטען געוועהד בײַ ד וואַר. די סאָלדאַטען 
האָבען אין האַנד אַריין— ריטער בעקומען…

שן צורי לאַטע שורות , מיט אַ שמאָלן דורכגאַנג אין 
דערמיט , האָט מע זיי אויסגעשטעל : „דורך אַ סטרוי" וועט מען 
טרייבען...

וועמען ז

עפּעט אַ יול מוט אַ צעריסען העמ.ד אויס'ן מדערן ליב , 
מיט אַ יאַרמולקע אויפ'ן האַלב־געגאָלטען קאָפ. אָט , 
פיהרט מען איהם צו.

פאַר וואָס קומ עס איהם ? ווער ווייסט , אַ האָרצייטישע זאַך 
! אפשר פאַר אַ געה , אשר פאַר אַ גזלה צי אַ רעיהת, און 
אפשר אַ בלטל... עס אין דאָך גשוען סאַ צייטען....

און ד זלנער שמייכלען און טראַכטען : נאָך וואָס האָט מען 
אונן אַעי פיעל גענומען און אויסגעשטעלט ? ער וועט נישט 
אויסהאַלטען ר העלט

נאָר אָט שטופט מען איהס צווישען די שורות אַריין , אָט 
געהט ער... און ער געהט גלייך און שטרויכעלט גישט און 
פאַלט ניש… ער לייזט שטיץ און האַלט זיי אויס...

אַ ציס־צדן אפט אָן דעמאָלט די סאָלדאַ ען ! ער געהט 
נאָך , ער געהט

און ד רטער פייפען אין דער לופט ווי די רושת און האַפּען 
דום שי לייב ווי ד לאַעען. און דאָס בלוט פֿון מאָגערן 
גוף שפּריצט און שפּריצט , און הערט נישט אַף צפּ 
שפּריצען !

הו—האַ ! דאָ—האַ !

אין טיטען טרעט אַ סאָלראַט צו הויך און וואַרפט דעם 
פער'משא'ן אַראָב די יאַרמאָלקע פֿון קאָפּ. נאָך עטליכע 
טריט בעמערקט ער עס , ער פער'משפט'ער... ער גיט זיך אַ 
ריהד , גליך ער האָט אַ ישוב , און איז זיך מישב , אין 
דרעהש זיך אום : ער וועט נישט געהן עלוי הראש , און ער 
געהט צוריק בין צום אָרט , וווּ די יאַרמאָלקע לועגט , 
ער בויגט זיך איין און דעיט זי אַף , און דרעהט זיך צוריק 
אויס און גערט ווייטער , רוהיג , רויט פערבלוטיגט , גאָר מיט 
דער יאָימאָלקע אויפ'ן קאָפּ. אַזוי גדש ער , בין ער 
פאַלט...

און אַ־ו ער איז געפעלע , אין צוגעלויגען ד נשמה און 
געהשפּט ד יאַרמלקקן , וואָס האָ אַזזי פיעל אומוימיגע 
ימיץ געקאָסט , אן אין אַרויף מיט איהר צו ער פורטקע שן 
הימל.

און די דרע מתּגה אין אויך אָעענומע גערדען !

גען ח צדקים האָבק זיך פאַר איהד געמיהט : ד טויערן מן 
גן־ק האָ ן זיך רהח גאָך די דריי מתּנות געעפענט

— מת שענע מתּנות , אוימעלש שנע... צו־ניץ קומען זיי נישט. 
גאַר גישט ל ממיל , אָבעה למ―אוימערליש .

\stopsection

\stopchapter
\stoptext

\stopcomponent
