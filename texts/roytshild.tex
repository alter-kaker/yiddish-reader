\startcomponent roytshild
\product reader

\starttext

\startchapter[
	title={ווען איך בין רויטשילד}][
	subtitle={אַ מאָנאָלאָג פֿון אַ כּתרילעווקער מלמד},
	author={שלום־עליכם}
]

 ―  ווען איך בין רויטשילד, ― האָט זיך צעלאָזט אַ כּתרילעווקער מלמד איינמאָל אין אַ דאָנערשטיק, בעת די רביצין האָט אים געמאָנט אַף שבת און ער האָט נישט געהאַט, ― אוי, ווען איך זאָל זײַן רויטשילד! טרעפֿט, וואָס איך טו? ראשית חכמה, פֿיר איך אײַן אַ מינהג, אַז אַ ווײַב זאָל תּמיד האָבן באַ זיך אַ דרײַערל, בכדי זי זאָל פֿאַרשפּאָרן דולן אַ ספּאָדיק, אַז סע קומט דער גוטער דאָנערשטיק און ס׳איז נישטאָ אַף שבת… והשנית, קויף איך אויס די שבתדיקע קאַפּאָטע, אָדער ניין ― דעם ווײַבס קעצענעם בורניס ― לאָז זי אויפֿהערן פּיקן אין קאָפּ אַרײַן, אַז ס׳איז איר קאַלט! און קויף אַוועק איך די דאָזיקע שטוב אינגאַנצן מיט אַלע דרײַ חדרים,  מיט דער קאַמער,  מיט דער שפּיזאַרני, מיטן קעלער, מיטן בוידעם, מיט הכּל בכּל מכּל פֿלעקל ― לאָז זי נישט זאָגן, אַז ס׳איז איר ענג; נאַ דיר אַוועק צוויי חדרים, קאָך דיר, באַק דיר, וואַש דיר, בראָק דיר, און לאָז מיך צורו, איך זאָל קענען קנעלן מיט מײַנע תּלמידים מיט אַ ריינעם קאָפּ! נישטאָ קיין דאגות פּרנסה, מע באַדאַרף נישט קלערן, וואו נעמט מען אַף שבת ― מחיה נפֿשות! די טעכטער אַלע חתונה געמאַכט, אַראָפּ אַ האָרב פֿון די פּלייצעס ― וואָס פֿעלט מיר? הייב איך מיך אָן אַרומקוקן אַביסל אַף דער שטאָט. דאָס ערשטע בין איך מנדר אַ נײַעם דאַך אַפֿן אַלטן בית־המדרש, לאָז אויפֿהערן קאַפּען אַפֿן קאָפּ, בשעת ייִדן דאַוונען; און, להבֿדיל, דאָס מרחץ בוי איך איבער אַפֿסנײַ. וואָרעם נישט הײַנט־מאָרגן ― עס וועט דאָרט מוזן זײַן אַן אומגליק, חס ושלום, טאָמער פֿאַלט דאָס אויס אַקוראַט בשעת נשים באָדן זיך. און וויבאַלד דאָס באָד, מוז מען שוין דעם הקדש אַוודאי צעוואַרפֿן און אנידערשטעלן אַ „ביקור־חולים“, אָבער טאַקע וואָס אַ ביקור־חולים הייסט, מיט בעטלעך, מיט אַ דאָקטאָר, מיט רפֿואות, מיט יײַכלעך אַלע טאָג פֿאַר די חולאים, ווי עס פֿירט זיך אין לײַטישע שטעט. און אַ „מושבֿ־זקנים“ שטעל איך אַוועק, אַלטע ייִדן לומדים זאָלן זיך נישט וואַלגערן אין בית־המדרש באַ דער הרובע, און אַ חבֿרה „מלביש־ערומים“, אָרעמע קינדער זאָלן נישט אַרומגיין, איך בעט איבער אײַער כּבוד, מיט די פּופּקעס אין דרויסן, און אַ חבֿרה „גמילות־חסדים“, אַז איטלעכער ייִד, סײַ אַ מלמד, סײַ אַ בעל־מלאָכה, סײַ אַ סוחר אפֿילו זאָל פֿאַרשפּאָרן צאָלן פּראָצענט, נישט דאַרפֿן פֿאַרמשכּונען דאָס העמד פֿונעם לײַב, און אַ חבֿרה „הכנסת־כּלה“, אַז וואו ערגעץ אַן אָרעם מיידל, נעבעך אַ דערוואַקסענע, זאָל מען זי אויסקליידן, ווי עס געהער צו זײַן, און חתונה מאַכן, און נאָך כּדומה אַזעלכע חבֿרות פֿיר איך אײַן באַ אונדז אין כּתרילעווקע… נאָר וואָס איז שייך עפּעס דווקא נאָר באַ אונדז אין כּתרילעווקע? אומעטום, וואו עס געפֿינען זיך נאָר אחינו בני ישׂראל, פֿיר איך אײַן אַזעלכע חבֿרות, אומעטום, אַף דער גאַנצער וועלט! און בכדי עס זאָל זיך פֿירן מיט אַ סדר, ווי עס געהער צו זײַן ― טרעפֿט, וואָס טו איך? מאַך איך אַף אַלע חבֿרות איין חבֿרה אַ גרויסע, אַ צדקה־גדולה־חבֿרה, וואָס גיט אַכטונג אַף אַלע חבֿרות, אַף אַלע ייִדן. דאָס הייסט, אַף דעם כּלל ישׂראל, אַז ייִדן זאָלן אומעטום האָבן פּרנסה און לעבן אין אַחדות און זאָלן זיצן אין די ישיבֿות און לערנען: חומש מיט רש״י, מיט גמרא, מיט תּוספֿות, מיט מהרש״א, מיט אַלע שבֿע חכמות און מיט אַלע שבעים לשון, און אַף אַלע ישיבֿות זאָל זײַן איין ישיבֿה אַ גרויסע, אַ ייִדישע אַקאַדעמיע, אין דער ווילנע געוויינטלעך, וואָס פֿון דאָרטן זאָלן אַרויסגיין די גרעסטע לומדים און חכמים אין דער וועלט, און אַלצדינג זאָל זײַן אומזיסט, „על חשבון הגבֿיר“, אַף מײַן קעשענע, און אַלצדינג זאָל זיך פֿירן מיט אַ סדר און מיט אַ פּלאַן, עס זאָל נישט זײַן קיין „גיב־מיר־נאַ־דיר־כאַפּ־לאַפּ“, און אַלע זאָלן אין זינען האָבן נאָר טובֿת־הכּלל!… און בכדי מע זאָל קענען טראָגן אַפֿן קאָפּ דעם „כּלל“ ― וואָס דאַרף מען? דאַרף מען באַוואָרענען דעם „פּרט“. און מיט וואָס קען מען באַוואָרענען דעם פּרט? געוויינטלעך, מיט פּרנסה; וואָרעם פּרנסה, הערט איר, דאָס איז דער רעכטער עיקר; אָן פּרנסה קען נישט זײַן קיין אַחדות; איבערן שטיקל ברויט, מישטיינס געזאָגט, איז איינער דעם אַנדערן יורד לחייו, קאַפּאַל יענעם קוילען, סמען, הענגען!… אַפֿילו די שׂונאי ישׂראל, אונדזערע המנס פֿון דער גאַנצער וועלט, מיינט איר, וואָס האָבן זיי צו אונדז? גאָרנישט. נאָר צוליב פּרנסה. זיי זאָלן האָבן פּרנסה, וואָלטן זיי גאָר אַזוי שלעכט נישט געווען. פּרנסה ברענגט צו קנאה, קנאה ברענגט צו שׂנאה, און דערפֿון, רחמנא ליצלן, נעמען זיך אַלע צרות אַף דער וועלט, אַלע אומגליקן, נישט דאָ געדאַכט, מיט אַלע רדיפֿות, מיט אַלע הריגות, מיט אַלע רציחות און מיט אַלע מלחמות…

אוי, די מלחמות, די מלחמות! דאָס איז, הערט איר, גאָר אַ שחיטה פֿאַר דער וועלט! ווען איך בין רויטשילד, מאַך איך אויס מלחמות, אָבער טאַקע לחלוטין אויס!

וועט איר דאָך פֿרעגן: וויִאַזוי? נאָר מיט געלט. דהיינו? איך וועל עס אײַך געבן צו פֿאַרשטיין מיט אַ שׂכל: למשל, צוויי מלוכות צעאַמפּערן זיך איבער אַ נאַרישקייט דאָרט, אַ שטיקל ערד, וואָס איז ווערט אַ שמעק טאַביקע; „טעריטאָריע“ הייסט עס באַ זיי. {\em די} מלוכה זאָגט, אַז די טעריטאָריע איז {\em איר} טעריטאָריע, און {\em די} מלוכה זאָגט: „ניין, ס'איז {\em מײַן} טעריטאָריע“. מששת ימי בראשית, הייסט עס, האָט גאָט באַשאַפֿן אָט דאָס שטיקל ערד פֿון איר כּבֿוד וועגן. קומט צוגיין אַ דריטע און זאָגט: „איר זייט ביידע בהמות; די טעריטאָריע איז אַלעמענס טעריטאָריע, {\em אַ רשות הרבים} הייסט עס“. הקיצור, טעריטאָריע אַהער, טעריטאָריע אַהין, מע „טעריטאָריעט“ אַזוי לאַנג, ביז מע הייבט אָן שיסן פֿון ביקסן און האַרמאַטן, און מענטשן קוילען זיך איינס דאָס אַנדערע ווי די שאָף, און בלוט, בלוט גיסט זיך אַזוי ווי וואַסער.

אָבער אַז איך קום צו זיי לכתּחילה און זאָג זיי: „שאַט, ברידערלעך, לאָזט אײַך דינען. אין וואָס גייט דאָ באַ אײַך, אייגנטלעך, דער גאַנצער סכסוך? מע פֿאַרשטייט דען נישט אײַער מיין? איר מיינט נישט אַזוי די הגדה, ווי די קניידלעך. טעריטאָריע איז דאָך באַ אײַך נאָר אַן אויסרייד; דער עיקר איז דאָך באַ אײַך יענע מעשׂה, פּעטי־מעטי, קאָנטריבוציע! און וויבאַלד לשון קאָנטריבוציע ― צו וועמען קומט מען אָן מיט אַ הלוואה? צו מיר, צו רויטשילדן הייסט עס. אַמער, ווייסט איר וואָס? נאַ דיר, ענגלענדער מיט די לאַנגע פֿיס און מיט די קעסטלדיקע הויזן, אַ מיליאַרד! נאַ דיר, נאַרישער טערק מיט דער רויטער יאַרמלקע, אַ מיליאַרד! נאַ דיר, מומע רייזל, אויך אַ מיליאַרד, ממה־נפֿשך, גאָט וועט אײַך העלפֿן, וועט איר מיר אָפּצאָלן מיט פּראָצענט, נישט חלילה קיין גרויסן פּראָצענט, פֿיר אָדער פֿינף לשנה, איך וויל אַף אײַך נישט רײַך ווערן“…

פֿאַרשטייט איר שוין? אי איך האָב געמאַכט אַ געשעפֿט, אי מענטשן הערן אַף צו קוילען איינס דאָס אַנדערע, ווי די אָקסן, אומזיסט און אומנישט. און וויבאַלד אויס מלחמות, הײַנט צו וואָס באַדאַרף מען דאָס כּלי־זין, מיטן חיל, מיט אַלע זיבעצן זאַכן, מיטן גאַנצן טאַרעראַם? אַף תּשעה נײַנציק כּפּרות! און וויבאַלד אויס כּלי־זין, אויס חיל, אויס טאַרעראַם, איז דאָך אויס שׂנאה, אויס קנאה, אויס טערק, אויס ענגלענדער, אויס פֿראַנצויז, אויס ציגײַנער, אויס ייִד, להבֿדיל ― די גאַנצע וועלט באַקומט גאָר דעמאָלט אַן אַנדער פּנים, ווי אין פּסוק שטייט באַ אונדז געשריבן: „{\em והיה}, און עס וועט זײַן,  {\em ביום ההוא},  דאָס הייסט, אַז משיח וועט קומען!“… (שטעלט זיך אָפּ).

און אפֿשר, האַ?… ווען איך בין רויטשילד, קען זײַן, אַז איך בין גאָר מבֿטל דאָס געלט. אויס געלט! וואָרעם, לאָמיר זיך נישט נאַרן, וואָס איז דען געלט? ― געלט איז דאָך, אייגנטלעך, נאָר אַ הסכּם, אַן אײַנגערעדטע זאַך, מע האָט גענומען אַ שטיקל פּאַפּיר, אַוועקגעשטעלט אַ צאַצקע און אָנגעשריבן: „טרי רובליאַ סערעבראָם“. געלט, זאָג איך אײַך, איז נישט מער ווי אַ יצר־הרע, אַ תּאווה אַזעלכע, איינע פֿון די גרעסטע תּאוות, {\em וואָס אַלע ווילן דאָס און קיינער האָט דאָס נישט}… אָבער אַז סע זאָל לחלוטין גאָר נישט זײַן קיין שום געלט אַף דער וועלט, וואָלט דאָך דער יצר־הרע נישט געהאַט וואָס צו טאָן, און די תּאווה וואָלט נישט געווען קיין תּאווה. איר פֿאַרשטייט, צי ניין? אַי וואָס? איז דאָך די קשיא, וווּ וואָלטן דעמאָלט נעמען ייִדן אַף שבת? (פֿאַרטראַכט זיך אַף אַ ווײַלע). איז דער תּירוץ: למאַי וווּ וועל איך {\em איצטער}  נעמען אַף שבת?… 
\stopchapter

\stoptext

\stopcomponent
