% Corrected by: Marc Trius
% Email: derpayatz@gmail.com
% Text ource: http://www.cs.uky.edu/~raphael/yiddish/searchSholem.cgi
% Date retrieved: 11/15/2019

\startcomponent oytser
\project reader

\starttext
\startchapter[title={דער אוצר}] [
	author={שלום־עליכם}
]

אַף יענער זײַט באַרג, הינטערן אַלטן בית־המדרש, געפֿינט זיך אַן אוצר.

אַזוי האָט מען געשמועסט באַ אונדז אין שטעטל.

נאָר קומען צום אוצר איז ניט אַזוי גרינג. אַז אַלע ייִדן אין שטעטל וועלן לעבן
בשלום, און וועלן זיך נעמען אַלע אים זוכן, דעמאָלט וועט מען אים געפֿינען.

אַזוי האָט מען געשמועסט באַ אונדז אין שטעטל.

און אַז אַלע ייִדן וועלן לעבן צופֿרידן, עס וועט נישט זײַן קיין קנאה באַ ייִדן,
קיין שׂנאה, קיין קריג, קיין לשון־הרע, קיין רכילות, און מע וועט זיך נעמען אַלע,
וועט מען אָפּזוכן דעם אוצר, און אַז נישט ― וועט ער אַרײַן טיף־טיף אין דער ערד
אַרײַן…

אַזוי האָט מען געשמועסט באַ אונדז אין שטעטל, און מע האָט אָנגעהויבן זיך צו
שפּאַרן און איבערשפּאַרן, צו אַמפּערן זיך און צו ווערטלען זיך, צו זידלען און צו
קריגן זיך, וואָס ווײַטער אַלץ מער, אַלץ שטאַרקער, און אַלץ איבערן אוצר. דער האָט
געזאָגט: ער דאַרף זײַן דאָ, דער האָט געזאָגט: דאָרטן, ― און מען האָט נישט
אויפֿגעהערט זיך צו שפּאַרן און איבערשפּאַרן, צו אַמפּערן זיך און צו ווערטלען
זיך, צו זידלען און צו קריגן זיך, וואָס ווײַטער אַלץ מערער, אַלץ שטאַרקער, און
אַלץ איבערן אוצר ― און דער אוצר… האָט געזונקען אַלץ טיפֿער און טיפֿער אין דער
ערד אַרײַן.
\stopchapter
\stoptext

\stopcomponent
