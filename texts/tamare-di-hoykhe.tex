% Corrected by: Marc Trius
% Email: derpayatz@gmail.com
% Text sources: 
% http://blogs.yiddish.forward.com/oyneg-shabes/193954
% https://www.yiddishbookcenter.org/sites/default/
% files/downloads/PT59_tall_tamare_sm.pdf
%
% Date retrieved: 04/01/2020

\startcomponent tamare-di-hoykhe
\project reader

\starttext

\startchapter[
	title={טאַמאַרע די הויכע}][
	author={אַבֿרהם קאַרפּינאָוויטש}
]



די צרות האָבן זיך אָנגעהויבן באַ בערטען די לוכטע. זי 
האָט געהאַט אַ הײַזל אַף יאַטקעווער גאַס. ס׳איז געווען 
אַן אײַנגעפֿירט אָרט. דאָס געשעפֿט איז געגאַנגען מקנא 
צו זײַן. דאָרט איז געווען תּמיד פֿול, צוליב טאַמאַרען. 
ווילנער בחורים האָבן זיך געקלעפּט צו איר. פֿרײַטיק צו 
נאַכט האָט יענע ניט געהאַט קיין פֿרײַע מינוט.

איז אָבער בערטע געווען זייער זשעדנע. געוואָלט אַרײַנכאַפּן 
די וועלט. האָט זי אָנגעהויבן שרײַען אַף טאַמאַרען, מאַכן 
איר דעם טויט. דאָ רעדט זי צו פֿיל, דאָ לאַכט זי ווען מע 
דאַרף ניט, און לאָזט אַרויס געסט. וואָלט טאַמאַרע פֿאַרשלונגען 
די אַבידעס, ווײַל פֿאָרט איז באַ בערטען אין הײַזל געווען 
ווייך און וואַרעם. האָט אָבער בערטע אָפּגעטאָן אין איינעם אַ 
פֿרײַטיק צו נאַכטס אַ זאַך, וואָס דאָס איז שוין געווען איבער 
דער גאַל. אַזוי פֿיל, אַז טאַמאַרע האָט איבערגעוואַרט דעם שבת 
און גלײַך אַף מאָרגן גענומען איר קופֿערל, אײַנגעפּאַקט די 
פּאָר קליידעלעך, מיט אַלע פֿאָרזעצונגען פֿון דעם ראָמאַן 
{\em רעגינע די שפּיאָנקע,} וואָס זי פֿלעגט קויפֿן יעדן 
פֿרײַטיק, און אַוועק פֿון בערטען.

מיידלעך פֿון פֿאַך האָבן געעצהט טאַמאַרען ניט אַוועקצוגיין. 
מע האָט שוין ניט איין אַבידע אַראָפּגעשלונגען פֿון די 
באַלעבאָסטעס. אָבער טאַמאַרע די הויכע האָט געהאַט איר 
אייגענעם האָנאָר, כאָטש קיין גרויסע דאַמע איז זי אין 
ווילנע ניט געווען. אין וואָלאָקומפּיע, דער שיינער געגנט 
הינטער ווילנע, איז זי אַף דאַטשע ניט געפֿאָרן, און באַ דעם 
„גרינעם שטראַל“ אין קאַפֿע איז זי ניט געזעסן.

טאָ פֿאַר וואָס, טאַקע, איז טאַמאַרע אַוועק פֿון בערטען? די זאַך 
איז געווען אַזוי.

פֿרײַטיק צו נאַכטס האָבן זיך נוהג געווען צו באַזוכן טאַמאַרען 
אַלע שטענדיקע קאַוואַליערן, ווי איצקע דער האָז, זרחקע דער 
בעזוך, סענדער דער זאַיִקע פֿון ייִדישע גאַס, מענדל דער 
פּאָזשאַרניק און נאָך און נאָך. אַלע פֿלעגן זיך אויסזעצן אַף אַ 
באַנק אין פֿירהויז און וואַרטן אַף זייער ריי. 
אַרומגעוואַשענע, אָפּגעראַזירטע, מיט ריינע העמדער, ווי עס 
פֿירט זיך לכּבֿוד שבת. צווישן זיי האָט זיך אויסגעטיילט 
הערשעלע דער טרעגער, דער קאַרליק, וואָס זײַנע פֿיס זענען 
נישט דערגאַנגען ביז צו דער פּאָדלאָגע. טאַמאַרע האָט אים 
תּמיד באַזונדער געליובעט, ווײַל ער האָט זיך ניט צוגעכאַפּט 
צו איר ווי צו דעם קוגל פֿון טשאָלנט. ער האָט איר קיינמאָל 
ניט פֿאַרטשעפּעט מיט אַ האַדקען וואָרט. ער האָט אויך געהאַט 
שיינע מאַניערן, אַפֿילו געהאַט שׂכל צו ברענגען אַ מאָל אַ 
זעקעלע צוקערקעס.

אַ מאַנצביל איז ער געווען, וואָס ווײַבער הרגענען זיך פֿאַר 
אַזאַ. לויט זײַנע כּוחות האָט ער געדאַרפֿט בלײַבן איבער נאַכט, 
אָבער יענער האָט קוים געהאַט צו באַצאָלן פֿאַר איין באַזוך. 
איז 
ער געלעגן נאָך דעם מיט איר, די קורצע פֿיס אויסגעצויגן 
ביז 
אירע קנעכלעך און געבעטן:

―טאַמאַרינקע, לאָמיך ליגן נאָך אַ רגע לעבן דיר. איך 
וויל שפּירן דעם טעם פֿון אַ חתונה־געהאַטן…

האָט טאַמאַרע קיין וואָרט ניט איבערגערעדט און 
אויסגענוצט הערשעלעס חלום אַף אָפּצוכאַפּן דעם אָטעם 
מיט אַ קאַלטער גלאָז טיי, וואָס זי האָט זיך געהאַט 
צוגעגרייט און ניט געקאָנט דערגיין צו דעם אַ גאַנצן 
אָוונט.

בערטע האָט דערזען ווי עטלעכע בחורים שטייען אַף פֿון 
באַנק און ווילן אַוועקגיין. זי האָט אויך געהערט ווי 
סענדער דער זאַיִקע עצהט, מע זאָל אַריבער צו 
אסתּרקען מיט די ברילן. יענע האָט געהאַט אַ הײַזל 
אַף זאַוואַלנע גאַס, לעבן האָלצמאַרק. ער האָט נאָך 
דערבײַ צוגעאיקעט:

―וואָאָס וועט מען, מע דאָ זיצן ווי, ווי הינער אַף, 
אַף די איי… אייער?…

און אַזוי ווי הערשעלע איז געלעגן לעבן טאַמאַרען און 
זיך געקוויקט מיט דער רגע, האָט בערטע די לוכטע 
אָנגעהויבן קלאַפּן מיטן פֿויסט אין דער טיר פֿון 
טאַמאַרעס אַלקער און זיך צעשריִען:

―הערשעלע! אַ שׂרפֿה דיר אין… דו ווייסט דאָך וואו!…

אַזאַ ווי דו דאַרף ניט קומען פֿרײַטיק צו נאַכטס, נאָר 
אין דער וואָכן! צעברעכסט מיר דאָס געשעפֿט!

טאַמאַרע האָט געשלעפּט דאָס קופֿערל צו דער חבֿרטע, 
לאהקע די שוואַרצע, וועלכע האָט געוואוינט אַף 
שקאַפּלערנע, הינטערן וואָקזאַל. אַפֿן וועג האָט זיך 
אַלץ אין איר געבונטעוועט קעגן בערטען. יענע האָט אין 
גאַנצן פֿאַרגעסן, אַז אַ מיידל פֿון דער בראַזשנע איז 
אויך אַ מענטש מיט האַרץ און געפֿיל, ווי מע זינגט אין 
ייִדישן טעאַטער.

זי האָט דאָך ניט געאַרבעט באַ רעדלען מצה, וואו דער אויוון 
ברענט און מע דאַרף וואָס גיכער פֿאַררוקן די סחורה. אַף 
וועמען האָט זי זיך געוואָרפֿן, די באַנדערשניצע? אַף 
הערשעלען, דעם סאַמע־עלנטן, וואָס האָט ניט געהאַט אין גאַנץ 
ווילנע צו וועמען זיך צוצוטוליען…

באַ לאהקען האָט טאַמאַרע געהאַט צײַט אַרײַנצוקוקן אין אַ ביכל. 
פֿון אַלע אין פֿאַך איז זי געווען די סאַמע גראַמאָטנע, אַפֿילו 
געבעטן ביכער אין דער ביבליאָטעק „מפֿיצי־השׂכּלה“, וואָס אַף 
זאַוואַלנע גאַס. אמת, זי האָט געמוזט אָנקומען צום 
ביבליאָטעקער קראַסני, ער זאָל איר זאָגן וואָס צו לייענען. 
דורך אים טאַקע איז זי געקומען צו אַ ביכל, אָנגעשריבן דורך 
איינעם אַ דיומאַ מיט דעם נאָמען {\em די קאַמעליען־דאַמע.} 
זי האָט געלייענט די מעשׂה פֿאַר איר חבֿרטע לאהקע די 
שוואַרצע, און ביידע האָבן זיך גוט אָנגעוויינט, דער עיקר 
באַם סוף, ווען די קאַמעליען־דאַמע, וואָס איז גאָרנישט 
געווען קיין דאַמע, נאָר געגעסן דאָס זעלבע ביטערע שטיקל 
ברויט פֿון וואַלגערן זיך אין פֿרעמדע בעטן, איז אַוועק 
שטאַרבן. ווען סיאָמקע קאַגאַן, דער רעפּאָרטער פֿון דער צײַטונג 
{\em ווילנער טאָג,} האָט פֿאַרטראַכט צו גרינדן אַ 
פּראָפֿעסיאָנעלן פֿאַראיין פֿון די ווילנער גאַסן־מיידלעך, האָט 
טאַמאַרע די ערשטע גענומען אַ וואָרט, זי האָט דאָס 
אויפֿגעהויבן 
די פֿראַגע וועגן איבערשעהען, וואָס די באַנדערס, די 
באלעבאטים פֿון די הײַזקעלעך, האָבן ניט געוואָלט אָנערקענען. 
פֿון דעם פּראָפֿעסיאָנעלן פֿאַראיין איז גאָרניט געוואָרן, ווי 
פֿון אַנדערע סיאָמקעס פּראָיעקטן, אָבער טאַמאַרען זענען פֿון 
יענע טעג איבערגעבליבן אויסלענדישע ווערטער ווי 
{\em עקספּלואַטאַציע,} {\em קלאַסן־באַוואוסטזײַן} און 
{\em גענעראַל־שטרײַק.}

\parsep

טאַמאַרע איז אַוועק פֿון פֿאַך. זי האָט געוואוינט באַ 
לאהקען, וואָס האָט געהאַט געדונגען אַ קאַמער אַף 
שקאַפּלערנע, ווײַטלעך פֿון שטאָט. לאהקע האָט געלעבט 
דערפֿון וואָס זי פֿלעגט ברענגען צו זיך צופֿעליקע געסט, 
עלטערע ייִדן פֿון ניט־הי. האָט זי געקאָנט אונטערהאַלטן 
די חבֿרטע מיט אַ טעלער זופּ און אַ גלאָז טיי. טאַמאַרע 
האָט זייער געליטן דערפֿון, געזוכט וואו צו פֿאַרדינען 
אַ גראָשן, אָבער קיין גרויסע זיבן זאַכן האָט זיך ניט 
געמאַכט. בערטע די לוכטע האָט זי דווקא גערופֿן צוריק, 
אָבער טאַמאַרע האָט געהאַט איר האָנאָר. דער איינציקער 
מיט וועמען זי האָט זיך געטראָפֿן, איז געווען הערשעלע 
דער טרעגער. פֿלעגט ער זיך טאַקע אָפּשטעלן מיט איר פֿון 
צײַט צו צײַט באַ לאהקען, אָבער פֿון הערשעלען האָט מען 
געקאָנט האָבן מער ליבשאַפֿט ווי פּרנסה.

סיאָמקע קאַגאַן האָט זייער געוואָלט העלפֿן טאַמאַרען. 
ער האָט אין איר געזען דעם קרבן פֿון פֿאַרשאָלטענעם 
קאַפּיטאַליזם, מיט וועלכן ער האָט אַזוי ביטער געקעמפֿט 
אַלע זײַנע יונגע יאָרן, האָט ער ווידער אַ מאָל 
געפּרוּווט פֿאַרווירקלעכן זײַן פּלאַן וועגן עפֿענען 
אין ווילנע אַ שול פֿאַר ליבע. וועגן אָט דעם פּראָיעקט 
האָט ער אָנגעשריבן אַ גרויסן אַרטיקל אין דער צײַטונג, 
אונטערגעשטראָכן די וויכטיקייט און דעם כּבֿוד וואָס 
אַזאַ שול, די ערשטע אין דער וועלט, וועט ברענגען 
ווילנע, און אָנגעגעבן אין נדן צוויי הויך־קוואַליפֿיצירטע 
פּעדאַגאָגן―טאַמאַרע שימעלסקי און לאה ברענער, באַוואוסט 
אין די פֿאַכקרײַזן מיט די נעמען „טאַמאַרע די הויכע“ און 
„לאהקע די שוואַרצע“. אַזוי האָט זיי באַקענט דעם לייענער 
סיאָמקע קאַגאַן אין זײַן אַרטיקל.

טאַמאַרע האָט ניט געהאַלטן דערפֿון. מע האָט שוין איין 
מאָל געפּרוּווט, אַפֿילו באַשטעלט אַ שילד, אָבער פֿון 
געשעפֿט האָט זיך אויסגעלאָזט אַ טײַך.

סיאָמקע האָט אָבער ניט נאָכגעלאָזט. ער איז אַוועק 
איבער דער שטאָט זוכן לערן־מאַטעריאַל. האָט ער אַזוי 
לאַנג גענישטערט אין קליינעם פֿונקס ביכערקראָם אַף 
דײַטשע גאַס, ביז ער האָט זיך געכאַפּט, אַז דווקא הײַקל 
לונסקי דער ביבליאָטעקער פֿון סטראַשונס ביבליאָטעק, 
קאָן אים העלפֿן. איז סיאָמקע אַוועק צו הײַקלען פֿרעגן 
וואָס יענער קאָן טאָן פֿאַר אים. איז הײַקל לונסקי 
צוגעגאַנגען צו אַ זײַטיקער שאַפֿע אין דער ביבליאָטעק 
און אַרויסגענומען פֿון דאָרטן אַ דין ביכעלע, אַ 
קונטרס, וואו עס זענען געווען זייער גענוי פֿאַרשריבן 
אלע תּקנות וועגן ריכטיק באַשלאָפֿן אַ ווײַב און וואָס 
צו טאָן אַף צופֿרידן שטעלן דעם מאַן.

הײַקל לונסקי האָט געגלעט דעם קונטרס, איבערגעדרוקט פֿון 
אַן אַלטן כּתבֿ־יד, וואָס האָט געדאַרפֿט האָבן אַ גוטע 
פּאָר הונדערט יאָר. דער נאָמען פֿון מחבר איז ניט 
געווען אָנגעגעבן. דער קונטרס האָט געהייסן {\em אַהבֿה 
בתּענוגים,} וואָס אַף פּראָסט ייִדיש הייסט דאָס: {\em די 
ליבע אין פֿאַרגעניגנס.}

סיאָמקע האָט גלײַך פֿאָרגעלייגט לונסקין איבערזעצן דעם 
קונטרס אַף ייִדיש. יענער האָט אים מבֿטל געמאַכט מיט 
אַ שמייכל אין זײַן געדיכטער באָרד, און געשעפּטשעט 
דערבײַ:

―אײַ, סיאָמע, דער לימוד, וואָס דו ווילסט אײַנפֿירן אין 
ווילנע, האָבן אונדזערע חכמים געבראַכט פֿאַר דעם עולם 
פֿריִער פֿון דיר. וועסט מיט דעם ניט אַנטדעקן אַמעריקע…

ס׳האָבן ניט געהאָלפֿן סיאָמקעס אַרגומענטן, אַז די 
גאַנצע זאַך קאָן זײַן אַזוינס און אַזעלכס. דאָס וועט 
מרעיש־עולם זײַן. אַ שול פֿאַר ליבע אין ווילנע, וואָס 
באַדינט זיך מיט לערן־מאַטעריאַל פֿון אָריגינעלע, 
ייִדישע מקורות, ממש פֿון דער רבנישער ליטעראַטור. הײַקל 
לונסקי, מיט זײַן אײַזערנעם געדולד, האָט אויסגעהערט 
סיאָמקען ביזן סוף און פֿאַרשטופּט דעם קונטרס צוריק אין 
דער שאַפֿע צווישן אַנדערע זעלטענע ספֿרים.

טאַמאַרע האָט זיך נישט פֿאַרלאָזט אַף סיאָמקען מיט 
זײַנע פֿאַנטאַזיעס און אָנגעהויבן זוכן וואָס צו טאָן. 
פֿריִער האָט זי זיך געוואָלט שטעלן אַפֿן מאַרק מיט אַ 
קאָשיק הינערישן דרויב, אָבער די 
מאַרק־ייִדענעס האָבן איר ניט צוגעלאָזט. צו פֿיל 
הענדלערקעס האָבן זיך צעשטעלט מיט זייער סחורה. איז זי 
אַוועק אין שחיטה־שטיבל פֿליקן עופֿות. דאָרטן האָבן די 
ווײַבער זי יאָ פֿײַן אויפֿגענומען. ס׳האָט לאַנג ניט 
געדויערט און טאַמאַרע האָט מיטגעבראַכט לאהקען די 
שוואַרצע צו דער אַרבעט. זענען זיי ביידע געזעסן, 
פֿאַרזונקען די פֿיס אין בערג הינערישע און גענדזענע 
פֿעדערן, און פֿון צײַט צו צײַט זיך דערמאָנט די 
אַמאָליקע יאָרן, ווען פֿון איין גאַסט האָט מען געהאַט 
מער ווי פֿון צען אָפּגעפֿליקטע הינער.

לאהקעס ייִנגל, עלינקע, וואָס זי האָט געהאַט פֿון אַ 
צופֿעליקן ייִדן און געטענהט, אַז דאָס איז געווען 
אליהו־הנבֿיא, איז אומגעלאָפֿן איבער דעם שחיטה־שטיבל 
און אַלע פֿליקערקעס האָבן געקליבן נחת פֿון אים.

איין זאַך האָט סיאָמקע יאָ אויפֿגעטאָן פֿאַר 
טאַמאַרען. ער האָט זי אַרײַנגעצויגן אין אַ קרײַזל, 
וואָס האָט זיך געזאַמלט געהיים אין אַן אייבערשטיבל 
אַף ייִדישע גאַס, אין ראַמײַלעס הויף. דאָרטן זענען 
געזעסן לאַנגע אָוונטן אַ פּאָר הויזן־נייער, עטלעכע 
שנײַדער־געזעלן, נייטערקעס, אַ הענטשקע־מאַכער, און אַ 
בחור מיט הונגעריקע אויגן האָט געלויבט פֿאַר זיי דאָס 
לעבן אין גרויסן ראַטן־פֿאַרבאַנד, אונטער דער זון פֿון 
דער סטאַלינישער קאָנסטיטוציע. אַ לעבן מקנא צו זײַן. 
אַלע אין קרײַזל האָבן געוואוסט וואָס טאַמאַרע איז 
פֿריִער געווען. דער דאַרער בחור האָט זי פֿאָרגעשטעלט 
ווי אַ געוועזענעם שקלאַף פֿון דער בורזשואַזער 
אָרדענונג.

טאַמאַרע די הויכע האָט געפֿליקט עופֿות, באַזוכט פֿון 
מאָל צום מאָל דעם קרײַזל און זיך געסטאַרעט זײַן 
צופֿרידן מיט איר לעבן. ס׳איז ניט געווען אַזוי לײַכט. 
אַ מאָל פֿלעגט זי אַ צי טאָן אָנטאָן צוריק דאָס רעקל 
מיטן טיפֿן שליץ, וואָס דורך אים האָט מען געזען אירע 
הויכע פֿיס, אַרויפֿציִען אַף זיך די אָפֿענע בלוזקע, 
נעמען אין האַנט דעם לאַקירענעם רידיקול, און זיך 
אַוועקשטעלן אַף סאַוויטשער גאַס, 
לעבן קינאָ {\em פּיקאַדילי,} זשמורען מיט אַן אויג; טאָמער 
גייט אָבער ווער פֿאַרבײַ פֿון די נײַע באַקאַנטע, וואָס 
וועלן זיי זאָגן? דאָרטן, אין דעם בוידעמשטיבל אַף 
ייִדישע גאַס האָט מען זי אויפֿגענומען ווי זי וואָלט 
גאָרנישט געדינט באַ בערטען די לוכטע. האָט זי זיך 
באַנוגנט מיט דעם שחיטה־שטיבל, און מיט נעמען אַ ביכל 
פֿון דער ביבליאָטעק.

האָט דער ביבליאָטעקער קראַסני אונטערגערוקט טאַמאַרען 
אין אײַלעניש אַ בינטל דערציילונגען. די זאַמלונג האָט 
געהייסן {\em פֿאָלקסטימלעכע געשיכטן,} פֿון אַ שרײַבער, אַ 
געוויסן פּרץ. האָט זי גלײַך געוואָלט בעטן עפּעס 
אַנדערש—אַ ראָמאַן, אַ ליבע מיט דועלן. אָבער דאָס 
איז געווען פֿרײַטיק, לייענער האָבן זיך געשטופּט און 
געאײַלט נעמען אַ בוך אַף שבת, האָט טאַמאַרע 
אַרײַנגעלייגט די געשיכטן אין קאָשיק און אַוועק אַף 
שקאַפּלערנע, צו לאהקען.

זענען ביידע חבֿטרעס געזעסן אין אָוונט באַם טיש, ווי 
צוויי אויסגעדינטע סאָלדאַטן, און געשוויגן. עלינקע איז 
שוין געשלאָפֿן. אַפֿן קאַמאָד האָבן געצאַנקט די 
שבת־ליכט, וואָס לאהקע האָט זיך געמאַכט אַ מינהג 
אָנצוצינדן זינט איר קדיש איז געבוירן געוואָרן. האַרבסט 
איז געווען. אַ דריבנער רעגנדל האָט געקלאַפּט אין 
נידעריקן פֿענצטער וואָס איז אַרויס צו אַ שטיקל גאָרטן, 
ווי דער שטייגער פֿון די הילצערנע שטיבער אין יענער 
געגנט. טאַמערע האָט געבלעטערט די {\em געשיכטן} און 
שווערע 
ספֿקותן האָבן געקרימט אירע פֿולע ליפּן. לאהקע האָט 
געקליבן די ברעקלעך חלה פֿון שבתדיקן טיטשעך, און 
געשוויגן. זי האָט געהאַט פֿאַראיבל אַף דער חבֿרטע, 
וואָס אַ גאַנצן אָוונט האָט זי מיט איר ניט אויסגערעדט 
קיין וואָרט און נאָר געווען פֿאַרנומען מיט איר בוך. 
האָט זי זיך ניט אײַנגעהאַלטן און אַ בורטשע געטאָן:

―לאָמיר אויך וויסן וואָס דו זוברעוועסט דאָרטן?

טאַמאַרע האָט אויפֿגעהויבן אַף לאהקען אירע קאַרע קאַצנאויגן, 
וואָס זייער בליק האָט אַזוי געוואַרעמט די אײַנגייער אין 
בערטע די לוכטעס געמאַכן, און זיך צעשמייכלט:

―דאָס זענען דערציילונגען שווער צו פֿאַרדײַען. מיט אַ 
סך העברעיִשע ווערטער. מע דאַרף זײַן אַ רבֿ דערצו.

―טאָ צו וואָס דאַרפֿסטו זיך ברעכן די ציין? זונטיק 
וועסטו בעטן אַן אַנדער בוך, מע זאָל אים קאָנען נעמען 
אין מויל אַרײַן.

טאַמאַרע האָט זיך אַ רגע פֿאַרטראַכט און דערנאָך זיך 
מודה געווען:

―איין מעשׂה האָב איך דווקא יאָ צעקײַט. זייער… זייער… 
אַזאַ… נו… מיט…

לאהקע האָט זיך אונטערגעשטופּט:

―זאָג שוין מיט אַהער, מיט אַהין, מיט וואָס?

―מיט אַ געדאַנק…

―לאָמיר הערן דעם געדאַנק.

―איך מוז דיר איבערלייענען…

לאהקע האָט ניט געהאַט קיין געדולד:

―פֿריִער דערצייל מיר אין וואָס עס גייט דאָרטן.

טאַמאַרע האָט זיך אָפּגעהוסט און אָנגעהויבן דערציילן:

―ס׳איז אַ מעשׂה מיט דרײַ מתּנות. אַ ייִד, וואָס איז 
געשטאָרבן און מע לאָזט אים נישט אַרײַן אין גן־עדן, ביז 
ער וועט נישט ברענגען דרײַ מתּנות אין הימל.

―ווער דאַרף אין הימל דרײַ מתּנות?

טאַמאַרע האָט זיך אויפֿגערעגט:

―אַז דו פֿרעגסט פֿערדישע פֿראַגעס, וועסטו בלײַבן אָן 
דער מעשׂה. אַ שרײַבער שרײַבט, איז הער זיך צו!

―שאַ, שאַ, זע וואָס דאָ טוט זיך. מע טאָר שוין ניט 
פֿרעגן אויך?

טאַמאַרע האָט אַ מאַך געטאָן מיט דער האַנט און ממשיך 
געווען:

―האָט ער געבראַכט אין הימל אַ מתּנה אַ שפּילקע…

לאהקע האָט מורא געהאַט צו פֿרעגן אַף וואָס דאַרף מען 
אין הימל אַ שפּילקע, האָט זי געשוויגן און טאַמאַרע 
האָט דערציילט ווײַטער:

―דער געשטאָרבענער ייִד האָט געבראַכט נאָך צוויי מתּנות—אַ 
זעקעלע ארץ־ישׂראל־ערד און אַ יאַרמלקע. אָבער די שפּילקע האָט 
מיך גענומען פֿאַרן האַרץ…

טאַמאַרע האָט פֿאַרמאַכט די {\em פֿאָלקסטימלעכע געשיכטן,} 
און אָנגעהויבן דערציילן:

―די זאַך איז געווען אַזוי. גלחים האָבן פֿאַרמישפּט צום 
טויט אַ רבֿס אַ טאָכטער, ווײַל זי איז אַרויס אַף דער 
גאַס ווען ס׳איז געגאַנגען אַ פּראָצעסיע פֿון קלויסטער. 
מיט יאָרן צוריק האָט אַ ייִד נישט געטאָרט גיין אין די 
גוייִשקע גאַסן. האָט מען זי, דעם רבֿס טאָכטער, 
צוגעבונדן פֿאַרן צאָפּ צום עק פֿון אַ ווילדן פֿערד, ער 
זאָל זי שלעפּן אין געלאַף איבערן בריק. האָבן די גלחים 
זי געפֿרעגט וואָס פֿאַר אַ בקשה האָט זי פֿאַרן טויט. 
האָט דעם רבֿס טאָכטער געבעטן אַ שפּילקע. מיט דער 
שפּילקע וואָס זי האָט באַקומען, האָט זי צוגעשפּיליעט 
איר קלייד צום לעבעדיקן פֿלייש פֿון איר פֿוס, כּדי מע 
זאָל ניט זען אירע ווײַבערישע ערטער בעת דאָס פֿערד 
וועט זיך יאָגן מיט איר דורכן גאַס, און דאָס קלייד וועט 
זיך אויפֿהייבן מיטן ווינט.

לאהקע האָט זיך געכאַפּט פֿאַרן קאָפּ:

―צום סאַמע פֿלייש?

―אָט דאָס וואָס דו הערסט…

לאהקע איז דאָס נישט געפֿעלן געוואָרן:

―אַ מענטש גייט צום טויט, האָט ער ניט קיין אַנדערע 
דאגות? וואָס איז שוין דער אונטערשייד?

טאַמאַרע האָט דערווידערט:

―אָט דאָס טאַקע איז דאָך דער געדאַנק. צום טויט איז זי 
געגאַנגען, אָבער אַ רבֿס אַ טאָכטער איז זי געבליבן, 
ניט הפֿקר…

לאהקע האָט געצויגן אַן אַקסל. דער ספֿק צי דעם רבֿס 
טאָכטער אין פּרצעס דערציילונג איז געווען גערעכט האָט 
איר פֿאַרצויגן די ליפּן.

אַזוי זענען ביידע חבֿרטעס, וואָס אַראָפּוואַרפֿן פֿון 
זיך דאָס קלייד פֿאַר פֿרעמדע אויגן איז געווען באַ זיי 
אַ וואָכעדיקע זאַך, אַוועק צו זייערע געלעגערס, יעדע 
מיט איר אייגענעם אויספֿיר.

\parsep

טאַמאַרע איז געזעסן אין שחיטה־שטיבל צווישן די פֿעדערן, 
ביז ס׳איז אויסגעבראָכן די מלחמה. די דײַטשן מיט די רוסן 
האָבן זיך צעטיילט מיט דער פּוילישער מדינה. אין ווילנע 
האָבן דעם נײַנצנטן סעפּטעמבער, יאָר נײַן און דרײַסיק, 
זיך באַוויזן אין אַ שיינעם פֿריהאַרבסטיקן פֿרימאָרגן 
אַף די גאַסן טאַנקען מיט רויטע, פֿינפֿעקיקע שטערן 
אַף די שטאָלענע באָרטן. טאַמאַרע האָט אָפּגעשאָקלט דעם פּוך פֿון 
פֿאַרטעך און געלאָפֿן מיט אַלעמען זען דעם וואונדער. דער 
גאַנצער קרײַזל פֿון ראַמײַלעס הויף האָט געדרייט אַרום די 
טאַנקען ווי נאָענטע מחותּנים אַף אַ לאַנג דערוואַרטער חתונה. 
די שׂימחה האָט אָבער לאַנג ניט געדויערט. אין גאַנצן אַ 
זעקס וואָך. דורך נאַכט האָבן זיך די טאַנקען, ווי 
אָנגעגעסענע טשערעפּאַכעס, זיך אַרויסגעשאַרט צוריק פֿון 
ווילנע. די רוסן האָבן אָפּגעגעבן די שטאָט אַ מתּנה 
פֿאַר די ליטווינער. די קלומפּעס, ווי מע האָט זיי גערופֿן אין
ווילנע, האָבן זיך געפֿרייט מיט דער מתּנה באַ אַכט חדשים, 
ביז די סאָוועטן האָבן באַשלאָסן קומען צוריק, כּדי צו 
באַפֿרײַען די ליטווינער פֿון אַלע דאגות צו פֿירן 
אַליין אַ מדינה. דאָס איז געשען דעם פֿופֿצנטן יוני, 
נײַצן פֿערציק.

ערשט דעמאָלט זענען אַלע, וואָס האָבן זיך געזאַמלט אין 
אייבערשטיבל אַף ייִדישע גאַס געקומען צום לעבן. שײַקע 
אייכער, וואָס האָט פֿאַרלוירן אַ האַנט אין שפּאַנישן 
בירגערקריג, איז געוואָרן דירעקטאָר פֿון די שטאָטישע 
אויטאָבוסן. לאַם, דער רייניקער פֿון מענער־קאַפּעליושן, האָט 
באַקומען די שטעלע פֿון קינאָ־פֿאַרוואַלטער. ברײַקע דער 
הויזן־נײער האָט איבערגענומען בערגערס זאַוואָד פֿון 
סאָדע־וואַסער. בכלל האָט ווילנע פֿאַרשמעקט מיט דער דיקטאַטור 
פֿון פּראָלעטאַריאַט.

אַף טאַמאַרען האָט מען געזאָגט עדות, אַז זי איז 
לינק־געשטימט ניט פֿון הײַנט, און די נײַע מאַכט האָט 
איר ארײַנגעזעצט אין אַ סקלאַד פֿון פֿלייש־קאָמבינאַט.

בעסער פֿון אַלעמען האָט אויסגעפֿירט לאהקע די 
שוואַרצע. זי האָט אונטערגעכאַפּט אַ לייטענאַנט פֿון 
דער רויטער אַרמיי, אַ וואוילער שייגעץ. ניט קיין גרויסער 
טרינקער. גלײַך האָט ער איר געקויפֿט אַ ווײַסן שאַל, 
אַפֿן שטייגער פֿון די לייטענאַנטסקע ווײַבער, און 
אַוועק מיט איר לעבן, טאַקע אַף שקאַפּלערנע.

דער שאַל איז געלעגן אַף לאהקעס לאָקענעס ווי שניי אַף 
אַ בערגל פֿאַרלאָשענע קוילן. ס׳איז געווען אַ 
פֿאַרגעניגן צו קוקן אַף איר. לאהקע איז געוואָרן אַן 
אייגענע צווישן די אַנדערע ווײַבער פֿון גאַרניזאָן. זי 
האָט שוין געהאַט אויפֿגעכאַפּט שטיקלעך רוסיש לשון אַף 
אַזוי פֿיל, אַז זי איז מסוגל געווען צו געבן פֿאַרשטיין 
אירע קומושקעס דעם אונטערשייד צווישן אַ 
לאַנגער נאַכטהעמד און אַ באַלקלייד. די אָנגעפֿאָרענע, 
נײַע מאַדאַמען זענען געווען זיכער, אַז מע קאָן אין אַ 
נאַכטהעמד קומען טאַנצן אין אָפֿיצירן־קלוב. לאהקע האָט 
זיך באַם לייטענאַנט אַ ביסל אויסגעריבן, אָנגענומען אַ 
שטיקל פֿאַסאָן. ווילנע האָט זי ניט דערקענט.

טאַמאַרע האָט זיך אויך געהאַט דורכגעוואָרפֿן מיט 
עטלעכע נאַטשאַלניקעס, אָבער ס׳האָט זיך ניט געקלעפּט. 
זי האָט געפֿונען אַ ווינקל אַף שאַוועלסקע גאַס, כּדי 
ניט צו שטערן דער חבֿרטע איר גליק. ס׳איז טאַקע געווען 
אַ גליק. אַ דאַנק אָט דעם לייטענאַנט, האָט זיך לאהקע 
מיט איר ייִנגל געראַטעוועט.

בעתן אָפּטרעטן פֿון ווילנע, אין איין און פֿערציקסטן 
יאָר, האָט דער לייטענאַנט אַרויפֿגעזעצט לאהקען מיט 
עלינקען אַף אַ מיליטערישער לאַסטמאַשין וואָס האָט זיי 
דערפֿירט טיף אין הינטערלאַנד. לאהקע האָט געוואָלט 
מיטנעמען טאַמאַרען, איז אָבער די אײַלעניש און די מהומה 
געווען צו גרויס. דער לייטענאַנט האָט געשריִען, 
געשאָלטן אין דער מאַמען אַרײַן, ווײַל יעדע רגע קאָן 
אָנקומען אַ דײַטשער דעסאַנט.

…און טאַמאַרע די הויכע? זי האָט ניט באַוויזן קיין 
סך צו לײַדן פֿון די דײַטשן. שוין מיטוואָך, דעם 
זיבעצנטן סעפּטעמבער 1941, אַ צען טעג נאָכן פֿאַרטרײַבן 
די ווילנער ייִדן אין געטאָ, איז זי געשטאַנען צווישן 
קנאַפּע דרײַצן הונדערט ייִדן אַפֿן פֿעלד פֿון 
פּאָנאַר―דאָס אָרט וואָס די דײַטשן האָבן אויסגעקליבן 
אויסצוהרגענען די ווילנער ייִדן, ניט ווײַט פֿון דער 
שטאָט.

טאַמאַרע האָט אויך געהאַט אַ באַזונדער מזל―מע האָט 
איר ניט געדאַרפֿט פֿירן אין געטאָ. זי איז שוין דאָרט 
געווען פֿריִער. שאַוועלסקע גאַס איז געלעגן אין האַרץ 
פֿון דעם תּחום־המושבֿ, באַשטימט פֿאַר די ווילנער ייִדן.

די ערשטע חדשים פֿון דעם דײַטשן שאַלטעווען אין ווילנע 
האָט זיך טאַמאַרע געדרייט דאָ, דאָרטן. אַ ביסל האָט זי 
אונטערגעהאַנדלט מיט דרויב און פֿון צײַט צו צײַט זיך 
אַרויסגעהאָלפֿן מיט הערשעלען דעם טרעגער. ביז זי איז 
אַרײַנגעפֿאַלן, ווי אין אַ קעסלגרוב, צווישן גזירות, 
פֿאַראָרדערונגען וועגן האָבן אַ געהעריקן צעטל, אַ שײַן 
פֿון די דײַטשן אַף בלײַבן לעבן.

איצט איז זי געשטאַנען מיט נאָך פֿרויען אַפֿן פֿעלד 
פֿון פּאָנאַר און געקוקט אין דער ווײַט. וועגן וועמען, 
וועגן וואָס האָט זי געקאָנט טראַכטן? אַליין איז זי 
דאָך געווען אַזוי עלנט, ניט געהאַט קיינעם. זי איז 
געווען אַן {\em אונטערגעוואָרפֿענע,} זיך געהאָדעוועט אין 
יתומים־הויז. געקומען צו איר ברויט דורך זײַן אַ
דערלאַנגערקע באַ זוסקע דעם פּראָפֿעסאָר, וואָס האָט 
געהאַט אַ קנײַפּע אַף קאָנסקע גאַס. דאָס האָט ער איר 
געגעבן צו פֿאַרשטיין, אַז מיט אירע פֿיס, אויב זי וועט 
שטיין אַף די ראָגן, קאָן זי מאַכן געלט.

\parsep

און אָט איז געקומען אַ באַפֿעל מע זאָל זיך אָויסטאָן. 
טאַמאַרע האָט זיך ניט אויסגעטאָן. זי האָט פֿאַרגראָבן 
די נעגל פֿון אירע פֿינגער אינעם ווייכן פֿלייש פֿון 
אירע פּוכקע אָרעמס און ניט געגעבן דעם מינדסטן סימן, 
אַז זי גייט אָפּשפּיליען אַ קנעפּל.

טאַמאַרע האָט ניט צו ליב געטאָן דעם דײַטש, וועלכער איז 
געשטאַנען קעגן איבער איר מיט זײַן מאַשין־געווער. העכער 
פֿון אים, אין איר פּערקאַלענעם קליידל מיט די רויטע און 
בלויע בלימעלעך, האָט זי אים געקוקט גלײַך אין די אויגן. 
און זיך ניט געמאַכט פֿון זײַנע געשרייען.

לאַנג האָט ער ניט געשריִען. אין איין רגע איז שוין 
טאַמאַרע געלעגן אַפֿן זאַמד. געטראָפֿן פֿון זײַן 
קוילן־סעריע.

\parsep

אַזוי איז אַוועק צום טויט טאַמאַרע די הויכע, דאָס 
לעצטע ייִדישע גאַסן־מיידל אין ווילנע.

אַ פֿרוי וואָס איז אַ פֿאַרוואונדיקטע אַרויסגעקראָכן 
באַ נאַכט פֿון גרוב און מצליח געווען זיך אומצוקערן אין 
געטאָ, האָט דאָס אַלץ שפּעטער דערציילט.

\stopchapter
\stoptext
\stopcomponent
