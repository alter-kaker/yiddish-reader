% Corrected by: Marc Trius
% Email: derpayatz@gmail.com
% Text source: https://ocr.yiddishbookcenter.org/
% Date retrieved: 11/21/2019

\startcomponent oyb־nisht־nokh־hekher
\product reader

\starttext
\startchapter[title={אויב נישט נאָך העכער}] [
	subtitle={אַ חסידישע דערציילונג},
	author={י. ל. פּרץ}
]


און דער נעמיראָווער פֿלעגט סליחות־צײַט יעדן פֿרימאָרגן נעלם ווערן, פֿאַרשווינדן!

מען פֿלעגט אים נישט זען אין ערגעץ: ניט אין שול, ניט אין ביידע בתי־מדרשים, נישט באַ
אַ מנין, און אינדערהיים אודאי און אודאי נישט. די שטוב איז געשטאַנען אָפֿן. ווער עס
האָט געוואָלט, איז אַרײַן און אַרויסגעגאַנגען; געגנבעט ביים רבין האָט מען {\em נישט}.
אָבער קיין לעבעדיק באַשעפֿעניש איז אין שטוב נישט געווען.

וואו קען זײַן דער רבי?

וואו זאָל ער זיין? מן הסתם אין הימל! ווייניק געשעפֿטן האָט אַ רבי פֿאַר ימים־נוראים צו
פֿאַרזאָרגן? ייִדעלעך, קיין עיןהרע, דאַרפֿן פּרנסה, שלום, געזונט, גוטע שידוכים, ווילן
גוט און פֿרום זײַן, און די זינד זײַנען דאָך גרויס, און דער שטן מיט זײַנע טויזנט אויגן
קוקט פֿון איין עק וועלט ביז דער צווייטער, און ער זעט, און פֿאַרקלאָגט, און מסרט… און
― ווער זאָל העלפֿן, אַז נישט דער רבי?

אַזוי האָט דער עולם געטראַכט.

איינמאָל איז אָבער אָנגעקומען א ליטוואַק, לאַכט ער! איר קענט דאָך די ליטוואַקעס: פֿון
מוסר־ספֿרים האַלטן זיי ווייניק, דערפֿאַר שטאָפּן זיי זיך אָן מיט ש״ס ופּוסקים. ווײַזט דער
ליטוואַק א בפֿירושע גמרא ― די אויגן שטעכט ער אויס. אפֿילו משה רבינו, ווײַזט ער, האָט
בײַם לעבן נישט געקאָנט אַרויף אַפֿן הימל, נאָר געהאַלטן זיך צען טפֿחים {\em אונטערן}
הימל! נו, גיי שפּאַר זיך מיט א ליטוואַק!

― וואו־דען קומט אַהין דער רבי?

― מיין דאגה ! ― ענטפֿערט ער און ציט מיט די אַקסל, און תּוך כדי דבור (וואָס א ליטוואַק
קען!) איז ער זיך מישב, צו דערגיין די זאַך.

\parsep

נאָך דעמזעלבן פֿאַרנאַכט, באַלד נאָך מעריב, גנבעט זיך דער ליטוואַק צום רבין אין חדר
אַרײַן, לייגט זיך אונטערן רבינס בעט און ליגט. ער מוז אָפּוואַרטן די נאַכט און זען,
וואו דער רבי קומט אַהין, וואָס ער טוט סליחות־צײַט.

אַן אַנדערער וואָלט אפֿשר אײַנגעדרעמלט און פֿאַרשלאָפֿן די צײַט; אַ ליטוואַק טוט זיך אן עצה:
לערנט ער זיך אויסנווייניג א גאַנצע מסכתא! איך געדענק שוין נישט, חולין צי נדרים!

פֿאַרטאָג הערט ער, ווי מען קלאַפּט צו סליחות.

דער רבי איז שוין נישט געשלאָפֿן אַ צײַט. ער הערט אים שוין א גאַנצע שעה קרעכצן.

ווער עס האָט געהערט דעם נעמיראָווער קרעכצן, ווייס, וויפֿיל צער פֿאַר כל ישראל, וויפֿיל
יסורים עס האָט געשטעקט אין זיין יעדן קרעכץ… די נשמה פֿלעגט אויסגיין, הערנדיק דאָס
קרעכצן! אַ ליטוואַק האָט דאָך אָבער אַן אײַזן האַרץ, הערט ער צו און ליגט זיך ווײַטער! דער
רבי ליגט זיך אויך: דער רבי, זאָל לעבן {\em אויפֿן בעט}, דער ליטוואַק {\em אונטערן}
בעט.

\parsep

דערנאָך הערט דער ליטוואַק, ווי די בעטן אין הויז הייבן אָן סקריפּען… ווי די בני־בית
כאַפּן זיך אַרויס פֿון די בעטן, ווי מען מורמלט א ייִדיש וואָרט, מען גיסט וואַסער אַף די
נעגל, עס קלאַפּן אַף און צו די טירן… דערנאָך איז דער עולם אַרויס פֿון שטוב, עס ווערט
ווײַטער שטיל און פֿינצטער, דורכן לאָדן שײַנט קוים אַרײַן א קליין ביסל ליכט פֿון דער
לבנה…

מודה האָט ער געווען, דער ליטוואַק, אַז ווען ער איז געבליבן איינער אַליין מיטן רבין,
איז אים באַפֿאַלן אַן אימה. די הויט איז שוין אים אויפֿגעלויפֿן פֿאַר שרעק, די וואָרצלען
פֿון די פּאות האָבן אים געשטאָכן אין די שלייפֿן ווי די נאָדלען.

אַ קלייניקייט: מיטן רבין, און סליחות־צײַט, פֿאַרטאָנ, אַליין אין שטוב!…

אַ ליטוואַק איז דאָך אָבער איינגעשפּאַרט, ציטערט ער ווי א פֿיש אין וואַסער און ליגט.―

\parsep

ענדלעך, שטייט דער רבי, זאָל לעבן, אויף…

פֿריער טוט ער, וואָס א ייִד דאַרף טאָן… דערנאָך גייט ער צו צו דעם קליידער־אַלמער און
נעמט ארויס א פּעקל… עס באַווײַזן זיך פּויערשע קליידער: לייוונטנע פּלודערן, גרויסע
שטיוול, א סיערמיענגע, א גרויס פֿוטערן היטל מיט א ברייטן לאַנגן לעדערנעם פּאַס,
אויסגעשלאָגן מיט מעשענע נעגעלעך.

דער דבי טוט עס אָן…

פֿון דער קעשענע, פֿון סיערמיענגע, שטאַרצט אַרויס אַן עק פֿון א גראָבן שטריק… פֿון א
פּויערשן שטריק!

דער רבי גייט אַרויס; דער ליטוואַק―נאָך!

דורכגייענדיק, טרעט אָפּ דער רבי אין קיך, בייגט זיך איין, פֿון אונטער א בעט נעמט ער
אַרויס א האַק, פֿאַרלייגט זי אונטערן פּאַס און גייט אַרויס פֿון שטוב.

דער ליטוואַק ציטערט, נאָר ער טרעט נישט אָפּ.

\parsep

א שטילער, ימים־נוראיםדיקער פּחד וויגט זיך איבער די טונקעלע גאַסן. אָפֿט רייסט זיך
אַרויס א געשריי פֿון סליחות פֿון ערגעץ א מנין, אָדער א קראַנקער קרעכץ פֿון ערגעץ א
פֿענצטער… דער רבי האַלט זיך אַלץ אין די זײַטן פֿון די גאַסן, אין די שאָטן פֿון די הײַזער…
פֿון איין הויז צום אַנדערן שווימט ער אַרויס און דער ליטוואַק נאָך אים…

און דער ליטוואַק הערט, ווי זײַן אייגן האַרץ־קלאַפּן מישט זיך צוזאַמען מיטן קול פֿון
רבינס שווערע טריט; אָבער גיין גייט ער, און קומט צוזאַמען מיטן רבין אויס דער שטאָט
אַרויס.

\parsep

הינטער דער שטאָט שטייט א וועלדל.

דער רבי, זאָל לעבן, נעמט זיך אין וועלדל אַריין. ער אכט

חסידיש

א דרייסיג־פֿערציג טויט און שטעלט זיך אָפּ ביי׳ן א ביימל און דער ליטוואַקך ווערט נבהל
ונשתומם, זעענדיק, ווי דער ריבי נעמט אַרויס פֿון פּאַס די האַק און שלאָגט אין בוימל
אַריין.

ער זעט, ווי דער רבי האַקט און האַקט, ער הערט ווי דאָס בוימל קרעכצט און קנאַקט. און
דאָס בוימל פֿאַלט, און דער רעי שפּאַלט עס אויף ליפּעס… די ליפּעס ― אויף דינע שייטלעך ;
און ער מאַכט זיך א בינטל האָלץ, נעמט עס אַרום מיט׳ן שטריקפֿון פֿון קעשענע, ער וואַרפֿט
דאָס בינטל האָלץ איבער די פּלייצעס, שטעקט צוריק אַריין די האַק אינ׳ם פּאַס, לאָזט זיך
אַרויס פֿון וואַלד און גייט צוריק אין שטאָט אַריין.

אין א הינטערגעסל שטעלט ער זיך אָפּ באַ אן אָרים, האַלב איינגעבראָכן הייזל און קלאַפּט אָן
אין פֿענצטערל.

― ווער איז ?―פֿרעגט מען דערשראָקן פֿון שטוב אַרויס. דער ליטוואַק דערקענט, אַז עס איז א
קול פֿון א אידענע, פֿון א קראַנקער אידענע.

― יאַ !―ענטפֿערט דער רבי אויף פּויעריש לשון.

― קטאָ יאַ ?ערעגט מען ווייטער פֿון שטוב.

און דער רבי ענשעדט ווייטער, אויף מאַלאָראָסיש לשון : וואַסיל !

― וואָס אר א וואַסיל און וואָס ווילסטו, וואַסיל ?

― האָלץ, ― זאָגט דער פֿאַרשטעלטער וואַסיל,―האָב איך צו פֿאַרקויפֿן ! זייער ביליג… בחצי
חנם האָלץ !

און ניט וואַרטנדיק אויף א תּשובה, נעמט ער זיך אין שטוב אַריין.

דער ליטוואַק גנב׳עט זיך אויך אַריין און, ביים גרויען ליכט פֿון פֿריען מאָרגן, זעט ער
אן אָרימע שטוב, צעבראָכן, אָרים כליבית… אין בעט ליגט א קראַנקע אידענע, פֿאַרוויקלט מיט
שמאַטעס, און זי זאָגט מיט א ביטער קול :

― קויפֿן ? מיט וואָס זאָל איך קויפֿן ? וואו האָב איך, אָרימע אלמנה, געלט ?

1

י. ל. פּרע

― איך וועל דיר באָרגן ! ― ענטפֿערט דער פֿאַרשטעלטער וואַסיל, אינגאַנצן זעקס גראָשן !

― פֿון וואַנען וועל איך דיר באַצאָלן ?―קרעכצט די אָרימע אידענע.

― נאַריש מענטש,―מוסר׳ט זי דער רבי,―זע, דו ביזט אן אָרימע קראַנקע אידענע און איך
געטרוי דיר דאָס ביסל האָלץ, איך בין במות. אַז דו וועסט באַצאָלן ; און דו האָסט אַזאַ
שטאַרקען, גרויסן גאָט, און געטרויסט אים נישט… און האָסט אויף אים אויף נאַרישע זעקס
גראָשן פֿאַר׳ן בינטל האָלץ, קיין בטחון נישט !

― און ווער וועט מיר איינהייצן ?―קרעכצט די אלמנה,― איך האָב דען כוח אויפֿצושטיין ?
מיין זון איז אויף דער אַרבעט געבליבן.

― איך וועל דיר איינהייצן אויך,―זאָגט דער רבי.

און, אַריינלייגנדיק דאָס האָלץ אין אויוון, האָט דער רבי קרעכצנדיק, געזאָגט סליחות דעם
ערשטן פּזמון…

און, אַז ער האָט עס אונטערגעצונדן, און דאָס האָלץ האָט פֿריילעך געברענט, האָט ער
געזאָגט, שוין א ביסל לוסטיגער, פֿון די סליחות דעם צווייטן פֿומוו…

דעם דריטן פּזמון האָט ער געזאָגט, ווען עס האָט זיך אויסגעברענט און ער האָט די בלעך
פֿאַרמאַכט…

דער ליטוואַק, וואָס האָט דאָס אַלץ געזען, איז שוין געבליבן א נעמיראָווער חסיד.

און שפּעטער, אויב א חסיד האָט אַמאָל דערציילט, אַז דער נעמיראָווער הויבט זיך אויף,
סליחות־צײַט, יעדן פֿרימאָרגן, און פֿליט אַרויף אין הימל אַריין, פֿלעגט שוין דער ליטוואַק
נישט לאַכן, נאָר צוגעבן שטילערהייט :

― אויב נישט נאָך העכער !
\stopchapter
\stoptext

\stopcomponent
