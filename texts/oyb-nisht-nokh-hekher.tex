% Corrected by: Marc Trius
% Email: derpayatz@gmail.com
% Text source: https://ocr.yiddishbookcenter.org/
% Date retrieved: 11/21/2019

\startcomponent oyb-nisht-nokh-hekher
\project reader

\starttext
\startchapter[title={אויב נישט נאָך העכער}] [
	subtitle={אַ חסידישע דערציילונג},
	author={י. ל. פּרץ}
]


און דער נעמיראָווער פֿלעגט סליחות־צײַט יעדן פֿרי\-מאָרגן נעלם ווערן, פֿאַרשווינדן!

מען פֿלעגט אים נישט זען אין ערגעץ: ניט אין שול, ניט אין ביידע בתי־מדרשים, נישט באַ
אַ מנין, און אינדערהיים אודאי און אודאי נישט. די שטוב איז געשטאַנען אָפֿן. ווער עס
האָט געוואָלט, איז אַרײַן און אַרויסגעגאַנגען; געגנבעט באַם רבין האָט מען {\em נישט}.
אָבער קיין לעבעדיק באַשעפֿעניש איז אין שטוב נישט געווען.

וואו קען זײַן דער רבי?

וואו זאָל ער זיין? מן הסתם אין הימל! ווייניק געשעפֿטן האָט אַ רבי פֿאַר ימים־נוראים
צו פֿאַרזאָרגן? ייִדעלעך, קיין עין־הרע, דאַרפֿן פּרנסה, שלום, געזונט, גוטע שידוכים, ווילן
גוט און פֿרום זײַן, און די זינד זענען דאָך גרויס, און דער שטן מיט זײַנע טויזנט אויגן
קוקט פֿון איין עק וועלט ביז דער צווייטער, און ער זעט, און פֿאַרקלאָגט, און מסרט… און
― ווער זאָל העלפֿן, אַז נישט דער רבי?

אַזוי האָט דער עולם געטראַכט.

איינמאָל איז אָבער אָנגעקומען אַ ליטוואַק, לאַכט ער! איר קענט דאָך די ליטוואַקעס: פֿון
מוסר־ספֿרים האַלטן זיי ווייניק, דערפֿאַר שטאָפּן זיי זיך אָן מיט ש״ס ופּוסקים. ווײַזט דער
ליטוואַק אַ בפֿירושע גמרא ― די אויגן שטעכט ער אויס. אפֿילו משה רבינו, ווײַזט ער, האָט
באַם לעבן נישט געקאָנט אַרויף אַפֿן הימל, נאָר געהאַלטן זיך צען טפֿחים {\em אונטערן}
הימל! נו, גיי שפּאַר זיך מיט אַ ליטוואַק!

― וואו־דען קומט אַהין דער רבי?

― מײַן דאגה! ― ענטפֿערט ער און ציט מיט די אַקסל, און תּוך כדי דבור (וואָס אַ ליטוואַק
קען!) איז ער זיך מישב, צו דערגיין די זאַך.

\parsep

נאָך דעמזעלבן פֿאַרנאַכט, באַלד נאָך מעריב, גנבעט זיך דער ליטוואַק צום רבין אין חדר
אַרײַן, לייגט זיך אונטערן רבינס בעט און ליגט. ער מוז אָפּוואַרטן די נאַכט און זען,
וואו דער רבי קומט אַהין, וואָס ער טוט סליחות־צײַט.

אַן אַנדערער וואָלט אפֿשר אײַנגעדרעמלט און פֿאַר\-שלאָפֿן די צײַט; אַ ליטוואַק טוט זיך אן
עצה: לערנט ער זיך אויסנווייניג אַ גאַנצע מסכתא! איך געדענק שוין נישט, חולין צי
נדרים!

פֿאַרטאָג הערט ער, ווי מען קלאַפּט צו סליחות.

דער רבי איז שוין נישט געשלאָפֿן אַ צײַט. ער הערט אים שוין אַ גאַנצע שעה קרעכצן.

ווער עס האָט געהערט דעם נעמיראָווער קרעכצן, ווייס, וויפֿיל צער פֿאַר כל ישראל, וויפֿיל
יסורים עס האָט געשטעקט אין זיין יעדן קרעכץ… די נשמה פֿלעגט אויסגיין, הערנדיק דאָס
קרעכצן! אַ ליטוואַק האָט דאָך אָבער אַן אײַזן האַרץ, הערט ער צו און ליגט זיך ווײַטער! דער
רבי ליגט זיך אויך: דער רבי, זאָל לעבן {\em אויפֿן בעט}, דער ליטוואַק {\em אונטערן}
בעט.

\parsep

דערנאָך הערט דער ליטוואַק, ווי די בעטן אין הויז הייבן אָן סקריפּען… ווי די בני־בית
כאַפּן זיך אַרויס פֿון די בעטן, ווי מען מורמלט אַ ייִדיש וואָרט, מען גיסט וואַסער אַף די
נעגל, עס קלאַפּן אַף און צו די טירן… דערנאָך איז דער עולם אַרויס פֿון שטוב, עס ווערט
ווײַטער שטיל און פֿינצטער, דורכן לאָדן שײַנט קוים אַרײַן אַ קליין ביסל ליכט פֿון דער
לבנה…

מודה האָט ער געווען, דער ליטוואַק, אַז ווען ער איז געבליבן איינער אַליין מיטן רבין,
איז אים באַפֿאַלן אַן אימה. די הויט איז שוין אים אויפֿגעלויפֿן פֿאַר שרעק, די וואָרצלען
פֿון די פּאות האָבן אים געשטאָכן אין די שלייפֿן ווי די נאָדלען.

אַ קלייניקייט: מיטן רבין, און סליחות־צײַט, פֿאַרטאָנ, אַליין אין שטוב!…

אַ ליטוואַק איז דאָך אָבער איינגעשפּאַרט, ציטערט ער ווי אַ פֿיש אין וואַסער און ליגט. ―

\parsep

ענדלעך, שטייט דער רבי, זאָל לעבן, אויף…

פֿריער טוט ער, וואָס אַ ייִד דאַרף טאָן… דערנאָך גייט ער צו צו דעם קליידער־אַלמער און
נעמט ארויס אַ פּעקל… עס באַווײַזן זיך פּויערשע קליידער: לייוונטנע פּלודערן, גרויסע
שטיוול, אַ סיערמיענגע, אַ גרויס פֿוטערן היטל מיט אַ ברייטן לאַנגן לעדערנעם פּאַס,
אויסגעשלאָגן מיט מעשענע נעגעלעך.

דער דבי טוט עס אָן…

פֿון דער קעשענע, פֿון סיערמיענגע, שטאַרצט אַרויס אַן עק פֿון אַ גראָבן שטריק… פֿון אַ
פּויערשן שטריק!

דער רבי גייט אַרויס; דער ליטוואַק―נאָך!

דורכגייענדיק, טרעט אָפּ דער רבי אין קיך, בייגט זיך איין, פֿון אונטער אַ בעט נעמט ער
אַרויס אַ האַק, פֿאַרלייגט זי אונטערן פּאַס און גייט אַרויס פֿון שטוב.

דער ליטוואַק ציטערט, נאָר ער טרעט נישט אָפּ.

\parsep

א שטילער, ימים־נוראיםדיקער פּחד וויגט זיך איבער די טונקעלע גאַסן. אָפֿט רייסט זיך
אַרויס אַ געשריי פֿון סליחות פֿון ערגעץ אַ מנין, אָדער אַ קראַנקער קרעכץ פֿון ערגעץ א
פֿענצטער… דער רבי האַלט זיך אַלץ אין די זײַטן פֿון די גאַסן, אין די שאָטן פֿון די
הײַזער…פֿון איין הויז צום אַנדערן שווימט ער אַרויס און דער ליטוואַק נאָך אים…

און דער ליטוואַק הערט, ווי זײַן אייגן האַרץ־קלאַפּן מישט זיך צוזאַמען מיטן קול פֿון
רבינס שווערע טריט; אָבער גיין גייט ער, און קומט צוזאַמען מיטן רבין אויס דער שטאָט
אַרויס.

\parsep

הינטער דער שטאָט שטייט אַ וועלדל.

דער רבי, זאָל לעבן, נעמט זיך אין וועלדל אַרײַן. ער מאכט אַ דרײַסיק־פֿערציק טריט און
שטעלט זיך אָפּ באַן אַ בוימל און דער ליטוואַק ווערט נבהל ונשתומם, זעענדיק, ווי דער
רבי נעמט אַרויס פֿון פּאַס די האַק און שלאָגט אין בוימל אַרײַן.

ער זעט, ווי דער רבי האַקט און האַקט, ער הערט ווי דאָס בוימל קרעכצט און קנאַקט. און
דאָס בוימל פֿאַלט, און דער רעי שפּאַלט עס אויף ליפּעס… די ליפּעס ― אויף דינע שײַטלעך;
און ער מאַכט זיך אַ בינטל האָלץ, נעמט עס אַרום מיטן שטריק פֿון קעשענע, ער וואַרפֿט
דאָס בינטל האָלץ איבער די פּלייצעס, שטעקט צוריק אַרײַן די האַק אינעם פּאַס, לאָזט זיך
אַרויס פֿון וואַלד און גייט צוריק אין שטאָט אַרײַן.

אין אַ הינטערגעסל שטעלט ער זיך אָפּ באַ אַן אָרעם, האַלב אײַנגעבראָכן הייזל און קלאַפּט אָן
אין פֿענצטערל.

― ווער איז? ― פֿרעגט מען דערשראָקן פֿון שטוב אַרויס. דער ליטוואַק דערקענט, אַז עס איז
א
קול פֿון אַ ייִדענע, פֿון אַ קראַנקער ייִדענע.

― יאַ ! ― ענטפֿערט דער רבי אויף פּויעריש לשון.

― קטאָ יאַ? פֿרעגט מען ווײַטער פֿון שטוב.

און דער רבי ענשפֿערט ווײַטער, אויף מאַלאָראָסיש לשון: וואַסיל!

― וואָס פֿאר אַ וואַסיל און וואָס ווילסטו, וואַסיל?

― האָלץ, ― זאָגט דער פֿאַרשטעלטער וואַסיל, ― האָב איך צו פֿאַרקויפֿן! זייער ביליק… בחצי
חנם האָלץ!

און ניט וואַרטנדיק אויף אַ תּשובה, נעמט ער זיך אין שטוב אַרײַן.

\parsep

דער ליטוואַק גנבעט זיך אויך אַרײַן און, באַם גרויען ליכט פֿון פֿריען מאָרגן, זעט ער
אן אָרעמע שטוב, צעבראָכן, אָרעם כלי־בית… אין בעט ליגט אַ קראַנקע ייִדענע, פֿאַרוויקלט
מיט שמאַטעס, און זי זאָגט מיט אַ ביטער קול:

― קויפֿן? מיט וואָס זאָל איך קויפֿן? וואו האָב איך, אָרעמע אלמנה, געלט?

― איך וועל דיר באָרגן ! ― ענטפֿערט דער פֿאַר\-שטעלטער וואַסיל, אינגאַנצן זעקס גראָשן!

― פֿון וואַנען וועל איך דיר באַצאָלן ?―קרעכצט די אָרעמע ייִדענע.

― נאַריש מענטש, ― מוסרט זי דער רבי, ― זע, דו ביסט אן אָרעמע קראַנקע ייִדענע און איך
געטרוי דיר דאָס ביסל האָלץ, איך בין {\em בטוח} אַז דו וועסט באַצאָלן; און דו האָסט אַזאַשטאַרקען, גרויסן גאָט, און געטרויסט אים נישט… און האָסט אַף אים אַף נאַרישע זעקס
גראָשן פֿאַרן בינטל האָלץ, קיין בטחון נישט!

― און ווער וועט מיר אײַנהייצן ? ― קרעכצט די אלמנה, ― איך האָב דען כוח אויפֿצושטיין?מײַן זון איז אַף דער אַרבעט געבליבן.

― איך וועל דיר אײַנהייצן אויך, ― זאָגט דער רבי.

\parsep

און, אַרײַנלייגנדיק דאָס האָלץ אין אויוון, האָט דער רבי קרעכצנדיק, געזאָגט סליחות דעם
{\em ערשטן} פּזמון…

און, אַז ער האָט עס אונטערגעצונדן, און דאָס האָלץ האָט פֿריילעך געברענט, האָט ער
געזאָגט, שוין אַ ביסל לוסטיקער, פֿון די סליחות דעם {\em צווייטן פּזמון…}

דעם דריטן פּזמון האָט ער געזאָגט, ווען עס האָט זיך אויסגעברענט און ער האָט די בלעך
פֿאַרמאַכט…

דער ליטוואַק, וואָס האָט דאָס אַלץ געזען, איז שוין געבליבן אַ נעמיראָווער חסיד.

און שפּעטער, אויב אַ חסיד האָט אַמאָל דערציילט, אַז דער נעמיראָווער הויבט זיך אויף,
סליחות־צײַט, יעדן פֿרימאָרגן, און פֿליט אַרויף אין הימל אַרײַן, פֿלעגט שוין דער ליטוואַק
נישט לאַכן, נאָר צוגעבן שטילערהייט:

― אויב נישט נאָך העכער!
\stopchapter
\stoptext

\stopcomponent
